\documentclass[alsotrans,beameroptions={aspectratio=169}]{beamerswitch}
\usepackage{fprog}

\title{Мутиращи операции}

\date{29 ноември 2023 г.}

\lstset{language=Scheme}

\begin{document}

\begin{frame}
  \titlepage
\end{frame}

\begin{frame}
  \frametitle{Мутиращи операции в Scheme}

  Мутиращите операции в Scheme позволяват въвеждането на \alert{странични ефекти}.\\[4ex]
  \pause
  \textbf{Преглед:}
  \begin{itemize}[<+->]
  \item \lst{set!} --- промяна на оценка, свързана със символ
  \item \lst{set-car!}, \lst{set-cdr!} --- промяна на компоненти на точкови двойки
  \item \lst{begin} --- последователност от действия
  \item \lst{open-input-file}, \lst{open-output-file} --- работа с файлове
  \item \lst{read}, \lst{write}, \lst{display} --- вход и изход
  \end{itemize}
\end{frame}

\begin{frame}[fragile]
  \frametitle{Промяна на оценка, свързана със символ (\tt{set!})}

  \begin{itemize}[<+->]
  \item \tta{(set!} <символ> <израз>\tta)
  \item Търси се <символ> във веригата от среди
    \begin{itemize}
    \item Ако бъде намерен, свързва се с оценката на <израз>
    \item В противен случай --- \alert{грешка!}
    \end{itemize}
  \item Примери:
    \begin{itemize}
    \item \lst{(define a 2)}\hspace{3em}\evalsto{a}2
    \item \lst{(set! a 5)}\hspace{4em}\evalsto{a}5
    \item \lst{(define (sum x)}\only<-.>{\tt{ (begin }}\lst{ (set! a (+ a x)) a}\only<-.>{\tt)}\tt)\pause
    \item \evalsto{(sum 10)}{15}
    \item \evalsto{(sum 10)}{25}
    \item \alert{губи се референциалната прозрачност!}
    \end{itemize}
  \end{itemize}
\end{frame}

\begin{frame}[fragile]
  \frametitle{Пример: текуща сметка}

\begin{lstlisting}
(define (make-account sum)
  (lambda (amount)
    (if (< (+ amount sum) 0)
        (display "Insufficient funds!\n")
        (set! sum (+ sum amount)))
    sum))
\end{lstlisting}
  \onslide<+->
  \begin{itemize}[<+->]
  \item \lst{(define account (make-account 100))}
  \item \evalsto{(account 20)}{120}
  \item \evalsto{(account -50)}{70}
  \item \evalstos{\tt{(account -150)}}{
      \begin{tabular}[t]l
        \tt{"Insufficient funds"}\\
        \tt{70}
      \end{tabular}}
  \end{itemize}
\end{frame}

\begin{frame}
  \frametitle{Промяна на компоненти (\tt{set-car!}, \tt{set-cdr!})}

  \begin{itemize}[<+->]
  \item \tta{(set-car!} <двойка> <израз>\tta)
  \item<.-> \tta{(set-cdr!} <двойка> <израз>\tta)
  \item Съответният компонент на <двойка> се променя да сочи към оценката на <израз>
  \item \textbf{Примери:}
    \begin{itemize}
    \item \lst{(define p (cons (cons 1 2) (cons 3 4)))}
    \item \lst{(set-car! p 7)}
    \item \evalstop{p}{(7 . (3 . 4))}
    \item \lst{(set-cdr! p '(5 3))}
    \item \evalstop{p}{(7 5 3)}
    \item \tt{(set-cdr! (cdr p) p)}
    \item \evalstop{p}{(7 5 7 5 7 5 ...)}
    \end{itemize}
  \end{itemize}
\end{frame}

\end{document}
