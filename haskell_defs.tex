\documentclass{beamer}
\usepackage{fprog}

\title{Синтаксис на функции}

\date{6 януари 2016 г.}

\begin{document}

\begin{frame}
  \titlepage
\end{frame}

%\includeonlyframes{current}

\section{Пазачи}

\begin{frame}[fragile]
  \frametitle{Разглеждане на случаи}
  Дефинниране на функции с разглеждане на случаи.
  \begin{itemize}
    \item{} <име> \{<параметър>\}\\
      \hspace{3ex} \{ \tta| <пазач> \tta= <израз> \}
      \pause
    \item{} <име> <параметър$_1$> <параметър$_2$> ... <параметър$_k$>\\
      \hspace{3ex} \tta| <пазач$_1$> \tta= <израз$_1$>\\
      \ldots\\
      \hspace{3ex} \tta| <пазач$_n$> \tta= <израз$_n$>\\
      \pause
  \item ако <пазач$_1$> е \tt{True} връща <израз$_1$>, а ако е \tt{False}:
  \item \ldots
  \item ако <пазач$_n$> е \tt{True} връща <израз$_n$>, а ако е \tt{False}:
  \item \alert{грешка!}
    \pause
  \item За удобство \tt{Prelude} дефинира \tt{otherwise = True}
  \end{itemize}
\end{frame}

\begin{frame}[fragile]
  \frametitle{Разглеждане на случаи --- примери}
\begin{verbatim}
fact n
  | n == 0    = 1
  | n > 0     = n * fact (n - 1)
  | n < 0     = 0
\end{verbatim}
\pause\vspace{1em}
\begin{verbatim}
grade x
  | x >= 5.5    = "Отличен"
  | x >= 4.5    = "Много добър"
  | x >= 3.5    = "Добър"
  | x >= 3      = "Среден"
  | otherwise   = "Слаб"
\end{verbatim}
\end{frame}


\section{Локални дефиниции}

\begin{frame}
  \frametitle{Локални дефиниции с \tt{let}}
  \begin{itemize}
  \item \tta{let} \{ <име> \tta= <израз> \}\\
    \tta{in} <тяло>
    \pause
  \item \tta{let} <име$_1$> \tta= <израз$_1$>\\
    \hspace{5ex}<име$_2$> \tta= <израз$_2$>\\
    \hspace{5ex}\ldots\\
    \hspace{5ex}<име$_n$> \tta= <израз$_n$>\\
    \tta{in} <тяло>
    \pause
  \item дефинициите <име$_i$> = <израз$_i$> се въвеждат едновременно само за оценката на <тяло>
  \item дефинициите може да са взаимно рекурсивни
  \item редът на дефинициите е без значение
  \end{itemize}
\end{frame}

\begin{frame}
  \frametitle{Примери за \tt{let}}
  
  \begin{itemize}
  \item \evalsto{
  \end{itemize}
\end{frame}

\begin{frame}
  \frametitle{Локални дефиниции с \tt{where}}
  \begin{itemize}
  \item{} <функция> \{ <параметри> \} = <тяло>\\
        \hspace{5ex} \tta{where} \{ <име> \tta= <израз> \}
        \pause
  \item{} <функция> <параметър$_1$> \ldots <параметър$_k$> = <тяло>\\
        \hspace{5ex} \tta{where} <име$_1$> \tta= <израз$_1$>\\
        \hspace{11ex} <име$_2$> \tta= <израз$_2$>\\
        \hspace{11ex}\ldots\\
        \hspace{11ex} <име$_n$> \tta= <израз$_n$>\\
        \pause
  \item дефинициите <име$_i$> = <израз$_i$> се въвеждат едновременно само за оценката на <тяло>
  \item дефинициите може да са взаимно рекурсивни, редът е без значение
  \end{itemize}
\end{frame}

\begin{frame}
  \frametitle{Примери за \tt{where}}
  
\end{frame}

\begin{frame}[fragile]
  \frametitle{Локални дефиниции --- пример}
\begin{verbatim}
area x1 y1 x2 y2 x3 y3 =
   let a = dist x1 y1 x2 y2
       b = dist x2 y2 x3 y3
       c = dist x3 y3 x1 y1
       p = (a + b + c) / 2
   in sqrt (p * (p - a) * (p - b) * (p - c))
   where dist u1 v1 u2 v2 = sqrt (du^2 + dv^2)
          where du = u2 - u1
                dv = v2 - v1
\end{verbatim}
\end{frame}

\begin{frame}
  \frametitle{Сравнения на \tt{let} и \tt{where}}
  
\end{frame}

\section{Двумерен синтаксис}

\begin{frame}
  \frametitle{Подравняване на дефинициите}

\end{frame}

\begin{frame}
  \frametitle{Алтернативен синтаксис}
  
\end{frame}

\section{Образци}

\begin{frame}
  \frametitle{Дефиниране с поредица от равенства}
  
\end{frame}

\begin{frame}
  \frametitle{Образци}
  
\end{frame}

\begin{frame}
  \frametitle{Примери}
  
\end{frame}

\begin{frame}
  \frametitle{Повторение на променливи}
  
\end{frame}

\end{document}
