\documentclass[alsotrans]{beamerswitch}
\usepackage{fprog}

\title{Кортежи и списъци}

\date{4--11 декември 2019 г.}

\lstset{language=Haskell,style=Haskell}

\begin{document}

\begin{frame}
  \titlepage
\end{frame}
\section{Кортежи}

\begin{frame}
  \frametitle{Кортежи (tuples)}

  Кортежите са наредени $n$-торки от данни от произволен тип.
  \begin{itemize}[<+->]
  \item \textbf{Примери:} \lst{(1, 2)}, \lst{(3.5,'A', False)}, \lst{(("square", (^2)), 1.0)}
  \item Тип кортеж от $n$ елемента: \tuple t
  \item Стойности: наредени $n$-торки от вида \tuple x, където $x_i$ е от тип $t_i$
  \item Позволяват ``пакетиране"' на няколко стойности в една
  \item Операции за наредени двойки:
    \begin{itemize}
    \item \lst{(,) :: a -> b -> (a,b)} --- конструиране на наредена двойка
    \item \lst{fst :: (a,b) -> a} --- първа компонента на наредена двойка
    \item \lst{snd :: (a,b) -> b} --- втора компонента на наредена двойка
    \end{itemize}
  \end{itemize}
\end{frame}

\begin{frame}
  \frametitle{Потребителски типове}

  \begin{itemize}[<+->]
  \item Типът \lst{(String, Int)} може да означава:
    \begin{itemize}
    \item име и ЕГН на човек
    \item продукт с описание и количество
    \item сонет на Шекспир и поредният му номер
    \end{itemize}
  \item Удобно е да именуваме типовете, за да означаваме смисъла им
  \item \tta{type} <конструктор> \tta= <тип>
    \begin{itemize}
    \item конструкторите са идентификатори, започващи с главна буква
    \end{itemize}
  \item \textbf{Примери:}
    \begin{itemize}
    \item \lst{type Student = (String, Int, Double)}
    \item \lst{type Point = (Double, Double)}
    \item \lst{type Triangle = (Point, Point, Point)}
    \item \lst{type Transformation = Point -> Point}
    \item \lst{type Vector = Point}
    \item \lst{addVectors :: Vector -> Vector -> Vector}
    \item<.-> \lst{addVectors v1 v2 = (fst v1 + fst v2, snd v1 + snd v2)}
    \end{itemize}
  \end{itemize}
\end{frame}

\begin{frame}
  \frametitle{Особености на кортежите}

  \begin{itemize}[<+->]
  \item \evalstoerrp{fst (1,2,3)}
    \begin{itemize}
    \item \lst{fst} и \lst{snd} работят само над наредени двойки!
    \end{itemize}
  \item \lst{((a,b),c)} $\neq$ \lst{(a,(b,c))} $\neq$ \lst{(a,b,c)}
  \item Няма специален тип кортеж от един елемент\ldots
  \item \ldots но има тип ``празен кортеж'' \lst{()} с единствен елемент \lst{()}
    \begin{itemize}
    \item в други езици такъв тип се нарича \lst{unit}
    \item използва се за означаване на липса на информация
    \end{itemize}
  \end{itemize}
\end{frame}

\begin{frame}[fragile]
  \frametitle{Образци на кортежи}

  Образец на кортеж е конструкция от вида \tuple p.\\\pause
  Пасва на всеки кортеж от точно $n$ елемента \tuple x, за който образецът $p_i$ пасва на елемента $x_i$.
  \onslide<+->
  \begin{itemize}[<+->]
  \item \lst{addVectors (x1, y1) (x2, y2) = (x1 + x2, y1 + y2)}
  \item \lst{fst (x,_) = x}
  \item<.-> \lst{snd (_,y) = y}
  \item \lst{getFN :: Student -> Int}
  \item<.-> \lst{getFN (_, fn, _) = fn}
  \item образците на кортежи могат да се използват за ``разглобяване'' на кортежи при дефиниция
  \item \lst{(x,y) = (3.5, 7.8)}
  \item \lst{let (_, fn, grade) = student in (fn, min (grade + 1) 6)}
  \end{itemize}
\end{frame}

\begin{frame}[fragile]
  \frametitle{Именувани образци}

  \begin{itemize}[<+->]
  \item намиране на студент с по-висока оценка
\begin{lstlisting}
betterStudent (name1, fn1, grade1) (name2, fn2, grade2)
 | grade1 > grade2 = (name1, fn1, grade1)
 | otherwise       = (name2, fn2, grade2)
\end{lstlisting}
  \item ами ако имахме 10 полета?
  \item удобно е да използваме \alert{именувани образци}
  \item{} <име>\tta@<образец> \onslide<+->
\lstset{escapechar=!}
\begin{lstlisting}
betterStudent s1@(_, _, grade1) s2@(_, _, grade2)
 | grade1 > grade2 = s1
 | otherwise       = s2
\end{lstlisting}
\lstset{escapechar=@}
  \end{itemize}
\end{frame}

\section{Списъци}

\subsection{Дефиниция и синтаксис}


\begin{frame}
  \frametitle{Списъци}

  \begin{definition}
    \begin{enumerate}
    \item Празният списък \lst{[]} е списък от тип \lst{[a]}
    \item Ако \tt h е елемент от тип \tt a и \tt t е списък от тип \lst{[a]} то \lst{(h : t)} е списък от тип \lst{[a]}
      \begin{itemize}
      \item \tt h --- глава на списъка
      \item \tt t --- опашка на списъка
      \end{itemize}
    \end{enumerate}
  \end{definition}
  \onslide<+->
  \begin{itemize}[<+->]
  \item списъкът е последователност с \alert{произволна дължина} от елементи от \alert{еднакъв тип}
  \item \lst{(:) :: a -> [a] -> [a]}  е \alert{дясноасоциативна} двуместна операция
  \item \lst{(1:(2:(3:(4:[]))))} $=$ \lst{1:2:3:4:[]} $\neq$ \tta{((((1:2):3):4):[])}
  \item \hlist a e по-удобен запис за $a_1$\tt{:(}$a_2$\tt:\ldots($a_n$\tt{:[])}\ldots\tt)
  \item \lst{[1,2,3,4]} $=$ \lst{1:[2,3,4]} $=$ \lst{1:2:[3,4]} $=$ \lst{1:2:3:[4]} $=$ \lst{1:2:3:4:[]}\
  \end{itemize}
\end{frame}

\begin{frame}
  \frametitle{Примери}

  \begin{itemize}[<+->]
  \item \typestop{[False]}{[Bool]}
  \item \ntypesp{["Иван"{}, 4.5]}
  \item \ntypesp{[1]:2}
  \item \typestop{[[1,2],[3],[4,5,6]]}{[[Int]]}
  \item \typestop{([1,2],[3],[4,5,6])}{([Int],[Int],[Int])}
  \item \ntypesp{[(1,2),(3),(4,5,6)]}
  \item \typestop{((1,2),(3),(4,5,6))}{((Int,Int),Int,(Int,Int,Int))}
  \item \typestop{[[]]}{[[a]]}
  \item \typestop{[]:[]}{[[a]]}
  \item \typestop{[1]:[[]]}{[[Int]]}
  \item \ntypesp{[]:[1]}
  \item \typestop{[[1,2,3],[]]}{[[Int]]}
  \item \ntypesp{[[1,2,3],[[]]]}
  \item \typestop{[1,2,3]:[4,5,6]:[[]]}{[[Int]]}
  \end{itemize}
\end{frame}

\begin{frame}
  \frametitle{Низове}

  \begin{itemize}[<+->]
  \item В Haskell низовете са представени като списъци от символи
  \item \lst{type String = [Char]}
  \item Всички операции над списъци важат и над низове
  \item \textbf{Примери:}
    \begin{overprint}
    \begin{itemize}
    \item \lst!['H', 'e', 'l', 'l', 'o' ] == "Hello"!
    \item \lst!'H':'e':'l':'l':'o':[] == "Hello"!
    \item \lst!'H':'e':"llo" == "Hello"!
    \item \lst!\ "" == [] :: [Char]!
    \item \ntypesp{[[1,2,3],{}"{}"]}
    \item \typestop{["12",['3'],[]]}{[String]}
    \end{itemize}
  \end{overprint}
  \end{itemize}
\end{frame}
\subsection{Работа със списъци}

\begin{frame}
  \frametitle{Основни функции за списъци}

  % head, tail, null, length
  \begin{itemize}[<+->]
  \item \lst{head :: [a] -> a} --- връща главата на (непразен) списък
    \begin{itemize}
    \item \evalstop{head [[1,2],[3,4]]}{[1,2]}
    \item \evalstoerr{head []}
    \end{itemize}
  \item \lst{tail :: [a] -> [a]} --- връща опашката на (непразен) списък
    \begin{itemize}
    \item \evalstop{tail [[1,2],[3,4]]}{[[3,4]]}
    \item \evalstoerr{tail []}
    \end{itemize}
  \item \lst{null :: [a] -> Bool} --- проверява дали списък е празен
  \item \lst{length :: [a] -> Int} --- дължина на списък
  \end{itemize}
\end{frame}

\begin{frame}[fragile]
  \frametitle{Образци и списъци}

  Много удобно е да използваме образци на списъци:
  \begin{itemize}[<+->]
  \item $p_h$\tt:$p_t$ --- пасва на всеки непразен списък $l$, за който:
    \begin{itemize}
    \item образецът $p_h$ пасва на главата на $l$
    \item образецът  $p_t$ пасва на опашката на $l$
    \end{itemize}
  \item \alert{Внимание:} обикновено слагаме скоби \lst{(h:t)}, понеже операцията \tt: е с много нисък приоритет
  \item \tt[$p_1$\tt, $p_2$\tt, \ldots\tt, $p_n$\tt] --- пасва на всеки списък от точно $n$ елемента \tt[$x_1$\tt, $x_2$\tt, \ldots\tt, $x_n$\tt], за който образецът $p_i$ пасва на елемента $x_i$
  \item \textbf{Примери:}
    \begin{itemize}
    \item     \lst{head (h:_) = h}
    \item     \lst{tail (_:t) = t}
    \item     \lst{null [] = True}
    \item<.-> \lst{null _  = False}
    \item     \lst{length []    = 0}
    \item<.-> \lst{length (_:t) = 1 + length t}
    \end{itemize}
  \end{itemize}
\end{frame}

\begin{frame}
  \frametitle{Случаи по образци (\tt{case})}
  \begin{itemize}[<+->]
  \item \tta{case} <израз> \tta{of} \{ <образец> \tta{->} <израз> \}$^+$
  \item \begin{tabular}[t]{@{}l@{ }l}
          \tta{case} <израз> \tta{of}&<образец$_1$> \tta{->} <израз$_1$>\\
                                     &\ldots\\
                                     &<образец$_n$> \tta{->} <израз$_n$>
        \end{tabular}
  \item ако <израз> пасва на <образец$_1$>, връща <израз$_1$>, иначе:
  \item<.-> \ldots
  \item<.-> ако <израз> пасва на <образец$_n$>, връща <израз$_n$>, иначе:
  \item<.-> \alert{Грешка!}
  \item Използването на образци в дефиниции всъщност е синтактична захар за конструкцията \tt{case}!
  \item \tt{case} може да се използва навсякъде, където се очаква израз
  \end{itemize}
\end{frame}

\begin{frame}
  \frametitle{Генератори на списъци}

  Можем да генерираме списъци от последователни елементи
  \begin{itemize}
  \item \tta[$a$\tta{..}$b$\tta] $\rightarrow$ \tta[$a$\tta, $a+1$\tta, $a+2$\tta,\ldots $b$\tta]
  \item \textbf{Пример:} \evalsto{[1..5]}{[1,2,3,4,5]}
  \item \textbf{Пример:} \evalsto{['a'..'e']}{\ "abcde"}
  \item Синтактична захар за \tt{enumFromTo from to}\\[1em]
    \pause
  \item \tta[$a$\tta, $a+\Delta x$ \tta{..} $b$\tta] $\rightarrow$ \tta[$a$\tta, $a+\Delta x$\tta, $a + 2\Delta x$\tta, \ldots \tta, $b'$\tta], където $b'$ е най-голямото число $\leq b$, за което $b' = a+k\Delta x$
  \item \textbf{Пример:} \evalsto{[1,4..15]}{[1,4,7,10,13]}
  \item \textbf{Пример:} \evalsto{['a','e'..'z']}{\ "aeimquy"}
  \item Синтактична захар за \tt{enumFromThenTo from then to}
  \end{itemize}
\end{frame}

\begin{frame}[fragile]
  \frametitle{Рекурсивни функции над списъци}
  \begin{itemize}[<+->]
  \item \lst{(++) :: [a] -> [a] -> [a]} --- слепва два списъка
    \begin{itemize}[<.->]
    \item  \evalsto{[1..3] ++ [5..7]}{[1,2,3,5,6,7]}
    \end{itemize}
  \item \lst{a ++ b = if null a then b else head a : tail a ++ b}
  \item \lst{reverse :: [a] -> [a]} ---  обръща списък
    \begin{itemize}[<.->]
    \item \evalsto{reverse [1..5]}{[5,4,3,2,1]}
    \end{itemize}
    \onslide<+->
\begin{lstlisting}
reverse a
 | null a    = a
 | otherwise = reverse (tail a) ++ [head a]
\end{lstlisting}
\item \lst{(!!) :: [a] -> Int -> a} --- елемент с пореден номер (от 0)
    \begin{itemize}[<.->]
    \item \evalsto{\ "Haskell" \!\! 2}{'s'}
    \end{itemize}
  \item \lst{elem :: }\tta{Eq a =>}\lst{ a -> [a] -> Bool} --- проверка за принадлежност на елемент към списък
    \begin{itemize}[<.->]
    \item \evalsto{3 `elem` [1..5]}{True}
    \end{itemize}
  \end{itemize}
\end{frame}

\subsection{Полиморфизъм}

\begin{frame}
  \frametitle{Полиморфни функции}

  %++, !!, reverse, elem, notElem
  Функциите \lst{head}, \lst{tail}, \lst{null}, \lst{length}, \lst{reverse} и операциите \lst{++} и \lst{!!} са \textbf{полиморфни}
  \begin{itemize}[<+->]
  \item работят над списъци с елементи от произволен тип \lst{[t]}
  \item \tt t се нарича \textbf{типова променлива}
  \item свойството се нарича \textbf{параметричен типов полиморфизъм}
  \item подобно на шаблоните в C++
  \item \alert{да не се бърка с \textbf{подтипов полиморфизъм}, реализиран с виртуални функции!}
  \item \lst{[]} е \textbf{полиморфна константа}
  \end{itemize}
\end{frame}

\begin{frame}[fragile]
  \frametitle{Класове от типове (typeclasses)}

  Функцията \lst{elem} има специални изисквания към типа на елементите на списъка: трябва да могат да бъдат сравнявани с \lst{==} или \lst{/=}
  \begin{itemize}[<+->]
  \item \lst!elem :: !\tta{Eq t =>}\lst{ t -> [t] -> Bool}
  \item \lst{Eq} е \alert{клас от типове}
  \item \lst{Eq} е класът на тези типове, за които има операции \lst{==} и \lst{/=}
    \begin{itemize}
    \item можем да си мислим за класовете от типове като за ``интерфейси''
    \end{itemize}
  \item \lst{Eq t} наричаме \alert{класово ограничение} за типа \tt t (class constraint)
  \item множеството от всички класови ограничения наричаме \alert{контекст}
  \item \alert{инстанция} на клас от типове наричаме всеки тип, за който са реализирани операциите зададени в класа
  \item инстанции на \lst{Eq} са:
    \begin{itemize}
    \item \lst{Bool}, \lst{Char}, всички числови типове (\lst{Int}, \lst{Integer}, \lst{Float}, \lst{Double})
    \item списъчните типове \lst{[t]}, за които \tt t е инстанция на \lst{Eq}
    \item кортежните типове \tt($t_1$\tt, \ldots\tt, $t_n$\tt), за които $t_i$ са инстанции на \lst{Eq}
    \end{itemize}
  \end{itemize}
\end{frame}

\begin{frame}
  \frametitle{Стандартни класове}

  Някои от по-често използваните класове на Haskell:
  \begin{itemize}[<+->]
  \item \lst{Eq} --- типове с равенство
  \item \lst{Ord} --- типове с (линейна) наредба
    \begin{itemize}[<.->]
    \item операциите \lst{==}, \lst{/=}, \lst{>=}, \lst{<=}, \lst{<}, \lst{>}
    \item специалната функция \lst{compare}, която сравнява два елемента и връща \lst{LT}, \lst{GT} или \lst{EQ} в зависимост от резултата
    \item функциите \lst{min} и \lst{max}
    \end{itemize}
  \item \lst{Show} --- типове, чиито елементи могат да бъдат извеждани в низ
    \begin{itemize}[<.->]
    \item функция \lst{show :: a -> String}
    \end{itemize}
  \item \lst{Read} --- типове, чиито елементи могат да бъдат въвеждани от низ
    \begin{itemize}[<.->]
    \item функция \lst{read :: String -> a}
    \end{itemize}
  \item \lst{Num} --- числови типове
  \item<.-> \lst{Integral} --- целочислени типове
  \item<.-> \lst{Floating} --- типове с плаваща запетая
  \item \alert{числата в Haskell са полиморфни константи!}
  \end{itemize}
\end{frame}

\subsection{Отделяне на списъци}

\begin{frame}
  \frametitle{Отделяне на списъци (list comprehension)}

  Отделянето на списъци е удобен начин за дефиниране на нови списъци чрез използване на дадени такива
  \begin{itemize}[<+->]
  \item \tta[ <израз> \tta| <генератор> \{\tta, <генератор>\} \{\tta, <условие>\} \tta]
  \item{} <генератор> е от вида <образец> \tta{<-} <израз>, където
    \begin{itemize}
    \item{} <израз> е от тип списък \lst{[a]}
    \item{} <образец> пасва на елементи от тип \tt a
    \end{itemize}
  \item{} <условие> е произволен израз от тип \lst{Bool}
  \item За всеки от елементите генериран от <генератор>, които удовлетворяват \alert{всички} <условие>, пресмята <израз> и натрупва резултатите в списък
  \end{itemize}
\end{frame}

\begin{frame}
  \frametitle{Примери за отделяне на списъци}

  \begin{itemize}[<+->]
  \item \evalstop{[ 2 * x | x <- [1..5] ]}{[2,4,6,8,10]}
  \item \evalstop{[ x^2 | x <- [1..10], odd x]}{[1,9,25,49,81]}
  \item \lst{[ fn | (\_, fn, grade) <- students, grade >= 3 ]}
  \item \lst{[ x^2 + y^2 | (x, y) <- vectors, x >= 0, y >= 0]}
  \item Ако имаме повече от един генератор, се генерират всички възможни комбинации от елементи (декартово произведение)
  \item \evalstopnl{[ x++(' ':y) | x<-["green","blue"],y<-["sky","grass"]]}{["green sky","green grass","blue sky","blue grass"]}
  \item \evalstopnl{[ (x,y) | x <- [1,2,3], y <- [5,6,7], x + y <= 8 ]}{[(1,5),(1,6),(1,7),(2,5),(2,6),(3,5)]}
  \item \textbf{Задача.} Да се генерират всички Питагорови тройки в даден интервал.
  \end{itemize}
\end{frame}

\section{Функции над списъци}

\begin{frame}
  \frametitle{Отрязване на списъци}

  \begin{itemize}[<+->]
  \item \lst{init :: [a] -> [a]} --- списъка без последния му елемент
    \begin{itemize}[<.->]
    \item \evalsto{init [1..5]}{[1,2,3,4]}
    \end{itemize}
  \item \lst{last :: [a] -> a} --- последния елемент на списъка
    \begin{itemize}[<.->]
    \item \evalsto{last "Haskell"}l
    \end{itemize}
  \item \lst{take :: Int -> [a] -> [a]} --- първите $n$ елемента на списък
    \begin{itemize}[<.->]
    \item \evalsto{take 4 "Hello, world\!"}{\ "Hell"}
    \end{itemize}
  \item \lst{drop :: Int -> [a] -> [a]} --- списъка без първите $n$ елемента
    \begin{itemize}[<.->]
    \item \evalsto{drop 2 [1,3..10]}{[5,7,9]}
    \end{itemize}
  \item \lst{splitAt :: Int -> [a] -> ([a],[a])}
    \begin{itemize}[<.->]
    \item \lst{splitAt n l = (take n l, drop n l)}
    \end{itemize}
  \end{itemize}
\end{frame}

\begin{frame}[fragile]
  \frametitle{Агрегиращи функции}

  % maximum, minimum, product, sum, and, or, concat
  \small
  \begin{itemize}[<+->]
  \item \lst{maximum :: Ord a => [a] -> a} --- максимален елемент
  \item<.-> \lst{minimum :: Ord a => [a] -> a} --- минимален елемент
  \item \lst{sum :: Num a => [a] -> a} --- сума на списък от числа
  \item<.-> \lst{product :: Num a => [a] -> a} --- произведение на списък от числа
  \item \lst{and :: [Bool] -> Bool} --- конюнкция на булеви стойности
  \item<.-> \lst{or :: [Bool] -> Bool} --- дизюнкция на булеви стойности
  \item \lst{concat :: [[a]] -> [a]} --- конкатенация на списък от списъци
  \item \textbf{Примери:}
    \begin{itemize}[<+->]
    \item \lst{[(sum l, product l)| l <- ll, maximum l == 2*minimum l]}
    \item \lst{and [ or [ mod x k == 0 | x <- row] | row <- matrix]}
    \end{itemize}
  \end{itemize}
\end{frame}

\subsection{Функции от по-висок ред}

\begin{frame}
  \frametitle{$\lambda$-функции}  \begin{itemize}[<+->]

  \item \tta{\textbackslash}\{ <параметър> \}$^+$ \tta{->} <тяло>
  \item \tta{\textbackslash} <параметър$_1$> \ldots <параметър$_n$> \tta{->} <тяло>
  \item анонимна функция с $n$ параметъра
  \item всеки <параметър$_i$> всъщност е образец
  \item параметрите са видими само в рамките на <тяло>
  \item \textbf{Примери:}
    \begin{itemize}
    \item \lst{id = \\x -> x}
    \item \lst{const = \\x y -> x}
    \item \evalsto{(\\x -> 2 * x + 1) 3}{7}
    \item \evalsto{(\\x l -> l ++ [x]) 4 [1..3]}{[1,2,3,4]}
    \item \evalsto{(\\(x,y) -> x^2 + y) (3,5)}{14}
    \item \evalsto{(\\f x -> f (f x)) (*3) 4}{36}
    \end{itemize}
  \item отсичането на операции може да се изрази чрез $\lambda$-функции:
    \begin{itemize}
    \item \tt(<операция> <израз>\tt) = \lst{\\x -> x} <операция> <израз>
    \item \tt(<израз> <операция>\tt) = \lst{\\x ->}  <израз> <операция> \tt x
    \end{itemize}
  \end{itemize}
\end{frame}

\begin{frame}
  \frametitle{Свойства на $\lambda$-функциите}

  \begin{itemize}[<+->]
  \item
    \begin{tabular}[t]{ll}
      &\tt\textbackslash $x_1\;x_2\;\ldots\;x_n$ \tt{->} <тяло>\\
      \eqv &\tt\textbackslash $x_1$ \tt{-> \textbackslash}$x_2$ \tt{->} \ldots \tt\textbackslash $x_n$ \tt{->}
             <тяло>
    \end{tabular}
  \item
    \begin{tabular}[t]{ll}
      &\lst{f x =} <тяло>\\
      \eqv&\lst!f = \\x ->! <тяло>
    \end{tabular}
  \item
    \begin{tabular}[t]{ll}
      &\lst{f x y =} <тяло>\\
      \eqv& \lst!f x = \\y ->! <тяло>\\
      \eqv& \lst!f = \\x y ->! <тяло>
    \end{tabular}
  \item
    \begin{tabular}[t]{ll}
      &\tt f $x_1 \ldots x_n$ \tt= <тяло>\\
      \eqv& \tt f $x_1 \ldots x_{n-1}$ \tt= \tt\textbackslash $x_n$ \tt{->} <тяло>\\
      \eqv& \ldots\\
      \eqv& \lst{f = \\}$x_1 \ldots x_n$\tt{ -> }<тяло>
    \end{tabular}
  \item
    \begin{tabular}[t]{ll}
      &\lst{\\x y -> f x y}\\
      \eqv& \lst{\\x -> f x}\\
      \eqv& \tt f
    \end{tabular}
  \end{itemize}
\end{frame}

\begin{frame}
  \frametitle{Трансформация (\tt{map})}

  \begin{itemize}[<+->]
  \item \lst{map :: (a -> b) -> [a] -> [b]}
  \item \lst{map f l = [ f x  | x <- l ]}
  \item \lst{map \_ [] = []}
  \item<.-> \lst{map f (x:xs) = f x : map f xs}
  \item \textbf{Примери:}
    \begin{itemize}
    \item \evalstop{map (^2) [1,2,3]}{[1,4,9]}
    \item \evalstop{map (\!\!1) [[1,2,3],[4,5,6],[7,8,9]]}{[2,5,8]}
    \item \evalstop{map sum [[1,2,3],[4,5,6],[7,8,9]]}{[6,15,24]}
    \item \evalstopnl{map ("a "++) ["cat","dog","pig"]}{["a cat","a dog","a pig"]}
    \item \evalstop{map (\\f -> f 2) [(^2),(1+),(*3)]}{[4,3,6]}
    \end{itemize}
  \end{itemize}
\end{frame}

\begin{frame}[fragile]
  \frametitle{Филтриране (\tt{filter})}

  \begin{itemize}[<+->]
  \item \lst{filter :: (a -> Bool) -> [a] -> [a]}
  \item \lst{filter p l = [ x | x <- l, p x ]}
  \item
\begin{lstlisting}
filter _ [] = []
filter p (x:xs)
 | p x       = x : rest
 | otherwise = rest
 where rest = filter p xs
\end{lstlisting}
  \item \textbf{Примери:}
    \begin{overprint}
    \begin{itemize}
    \item \evalstop{filter odd [1..5]}{[1,3,5]}
    \item \evalstop{filter (\\f -> f 2 > 3) [(^2),(+1),(*3)]}{[(^2),(*3)]}
    \item \evalstopnl{map (filter even) [[1,2,3],[4,5,6],[7,8,9]]}{[[2],[4,6],[8]]}
    \item \evalstops{\lst{map (\\x -> map (\\f -> filter f x) [(<0),(==0),(>0)])} \tt{[[-2,1,0],[1,4,-1],[0,0,1]]}\\}{\tt{[[[-2],[0],[1]],[[-1],[],[1,4]],[[],[0,0],[1]]]}}
    \end{itemize}
  \end{overprint}
  \end{itemize}
\end{frame}

\begin{frame}
  \frametitle{Отделяне на списъци с \tt{map} и \tt{filter}}

  Отделянето на списъци е синтактична захар за \lst{map} и \lst{filter}\pause
  \small
  \begin{itemize}[<+->]
  \item \begin{tabular}[t]{l}
      \tt[<израз> \tt| <образец> \tt{<-} <списък>\tt, <условие>\tt]\\
      $\longleftrightarrow$\\
      \lst{map (\\}<образец> \tt{->} <израз>\tt)\\
      \hspace{5ex}\lst{(filter (\\}<образец> \tt{->} <условие>\tt) <списък>\tt)
    \end{tabular}
  \item \begin{tabular}[t]{l}
      \tt[<образец> \tt| <образец> \tt{<-} <списък>\tt,<условие$_1$>,<условие$_2$>\tt]\\
      $\longleftrightarrow$\\
      \lst{filter (\\}<образец> \tt{->} <условие$_2$>\tt)\\
      \hspace{5ex}\lst{(filter (\\}<образец> \tt{->} <условие$_1$>\tt) <списък>\tt)
    \end{tabular}
  \item \begin{tabular}[t]{l}
      \tt[<израз>\tt|<образец$_1$> \tt{<-} <списък$_1$>\tt,<образец$_2$> \tt{<-} <списък$_2$>\tt]\\
      $\longleftrightarrow$\\
      \lst{concat (map (\\}<образец$_1$> \tt{->}\\
      \hspace{15ex}\lst{map (\\} <образец$_2$> \tt{->} <израз>\tt) <списък$_2$>\tt)\\
      \hspace{10ex} <списък$_1$>\tt)
    \end{tabular}
  \end{itemize}
\end{frame}

\begin{frame}<1-14>[fragile]
  \frametitle{Дясно свиване (\tt{foldr})}

  \begin{itemize}[<+->]
  \item \lst{foldr :: (a -> b -> b) -> b -> [a] -> b}
  \item \lst{foldr op nv} \hlist x = \\
    $x_1$ \tt{`op` (}$x_2$ \tt{`op`} \ldots \tt($x_n$\tt{ `op` nv)} \ldots\tt)
  \item
\begin{lstlisting}
foldr _  nv [] = nv
foldr op nv (x:xs) = x `op` foldr op nv xs
\end{lstlisting}
  \item \textbf{Примери:}
    \begin{itemize}
    \item \lst{sum = foldr (+) 0}
    \item \lst{product = foldr (*) 1}
    \item \lst{concat = foldr (++) []}
    \item \lst{and = foldr (&&) True}
    \item \lst{or = foldr (||) False}
    \item \lst{map f = foldr (\x} \tt{\only<-10>{ r} -> \alt<10>{f x : r}{(f x:)}) []}
      \onslide<+->
    \item \footnotesize\lst{filter p = foldr (\x}\tt{\only<-13>{ r} -> \temporal<13>{if p x then x:r else r}{(if p x then (x:) else id) r}{if p x then (x:) else id}) []}
    \end{itemize}
  \end{itemize}
\end{frame}

\begin{frame}[fragile]
  \frametitle{Ляво свиване (\tt{foldl})}

  \begin{itemize}[<+->]
  \item \lst{foldl :: (b -> a -> b) -> b -> [a] -> b}
  \item \lst{foldl op nv }\hlist x = \\
    \tt(\ldots\tt{((nv `op` }$x_1$\tt{) `op` }$x_2$\tt) \ldots \tt{) `op` }$x_n$
  \item
\begin{lstlisting}
foldl _  nv [] = nv
foldl op nv (x:xs) = foldl op (nv `op` x) xs
\end{lstlisting}
  \item \textbf{Пример:}
    \begin{itemize}[<.->]
    \item \lst{flip f x y = f y x}
    \item \lst{reverse = foldl (flip (:)) []}
    \end{itemize}
  \end{itemize}
\end{frame}

\begin{frame}[fragile]
  \frametitle{Свиване на непразни списъци (\tt{foldr1} и \tt{foldl1})}

  \begin{itemize}[<+->]
  \item \lst{foldr1 :: (a -> a -> a) -> [a] -> a}
  \item \lst{foldr1 op }\hlist x = \\
    $x_1$\tt{ `op` (}$x_2$\tt{ `op` }\ldots \tt($x_{n-1}$\tt{ `op` }$x_n$\tt) \ldots\tt)
  \item
\begin{lstlisting}
foldr1 _  [x] = x
foldr1 op (x:xs) = x `op` foldr1 op xs
\end{lstlisting}
  \item \lst{foldl1 :: (a -> a -> a) -> [a] -> a}
  \item \lst{foldl1 op }\hlist x = \\
    \tt(\ldots\tt{((}$x_1$\tt{ `op` }$x_2$\tt) \ldots \tt{) `op` }$x_n$
  \item \lst{foldl1 op (x:xs) = foldl op x xs}
  \item \textbf{Примери:}
    \begin{itemize}
    \item \lst{maximum = foldr1 max}
    \item \lst{minimum = foldr1 min}
    \item \lst{last = foldl1 (\_ x -> x)}
    \end{itemize}
  \end{itemize}
\end{frame}

\begin{frame}
  \frametitle{Сканиране на списъци (\tt{scanl}, \tt{scanr})}

  \lst{scanr} и \lst{scanl} връщат историята на пресмятането на \lst{foldr} и \lst{foldl}
  \pause\small
  \begin{itemize}[<+->]
  \item \lst!scanr :: (a -> b -> b) -> b -> [a] ->! \tta{[b]}
  \item \lst{scanr op nv} \hlist x = \\
    \tt[$x_1$\tt{ `op` (}$x_2$\tt{ `op` }\ldots \tt($x_n$\tt{ `op` nv)} \ldots\tt{),}\\
    $x_2$\tt{ `op` (}\ldots \tt($x_n$\tt{ `op` nv)} \ldots\tt{),}\\
    \ldots\\
    $x_n$\tt{ `op` nv,}\\
    \tt{ nv]}
  \item \lst!scanl :: (b -> a -> b) -> b -> [a] ->! \tta{[b]}
  \item \lst{scanl op nv} \hlist x = \\
    \tt{[nv,}\\
    \tt{ nv `op` }$x_1$\tt,\\
    \tt{ (nv `op` }$x_1$\tt{) `op` }$x_2$\tt,\\
    \ldots\\
    \tt{(\ldots((nv `op` $x_1$) `op` $x_2$) \ldots ) `op` $x_n$]}
  \end{itemize}
\end{frame}

\begin{frame}
  \frametitle{Съшиване на списъци (\tt{zip}, \tt{zipWith})}

  \begin{itemize}[<+->]
  \item \lst{zip :: [a] -> [b] -> [(a,b)]}
    \begin{itemize}
    \item \evalstos{\lst{zip} \hlist x \hlist y}{\tt[($x_1$\tt, $y_1$\tt{), (}$x_2$\tt, $y_2$\tt{), }\ldots\tt{, (}$x_n$\tt, $y_n$\tt{)]}}
    \item ако единият списък е по-къс, спира когато свърши той
  \end{itemize}
  \item \lst{unzip :: [(a,b)] -> ([a],[b])}
    \begin{itemize}
    \item разделя списък от двойки на два списъка с равна
      дължина
    \item \evalstos{\lst{unzip}
        \tt{[(}$x_1$\tt,$y_1$\tt{), (}$x_2$\tt,$y_2$\tt{), }\ldots\tt{, (}$x_n$,$y_n$\tt{)]}}{\tt(\hlist x\tt, \hlist y\tt)}
    \end{itemize}
  \item \lst{zipWith :: (a -> b -> c) -> [a] -> [b] -> [c]}
    \begin{itemize}
    \item ``съшива'' два списъка с дадена двуместна операция
    \item \evalstos{\lst{zipWith op} \hlist x \hlist y}{\tt[$x_1$\tt{ `op` }$y_1$\tt, $x_2$\tt{ `op` } $y_2$\tt, \ldots\tt, $x_n$\tt{ `op` }$y_n$]}
    \end{itemize}
  \item \textbf{Примери:}
    \begin{itemize}
    \item \evalsto{zip [1..3] [5..10]}{[(1,5),(2,6),(3,7)]}
    \item \evalsto{zipWith (*) [1..3] [5..10]}{[5,12,21]}
    \item \lst{zip = zipWith (,)}
    \item \lst{unzip = foldr (\\(x,y) (xs,ys) -> (x:xs,y:ys)) ([],[])}
    \end{itemize}
  \end{itemize}
\end{frame}

\begin{frame}
  \frametitle{Разбивания на списъци}

  \begin{itemize}[<+->]
  \item \lst{takeWhile :: (a -> Bool) -> [a] -> [a]}
      \begin{itemize}
      \item връща първите елементи на списъка, за които е вярно условието
      \item \lst{takeWhile p = foldr (\\x r -> if p x then x:r else []) []}
      \item \evalsto{takeWhile (<0) [-3..3]}{[-3,-2,-1]}
      \end{itemize}
  \item \lst{dropWhile :: (a -> Bool) -> [a] -> [a]}
      \begin{itemize}
      \item премахва първите елементи на списъка, за които е вярно условието
      \item \evalsto{dropWhile (<0) [-3..3]}{[0,1,2,3]}
      \end{itemize}
  \item \lst{span :: (a -> Bool) -> [a] -> ([a], [a])}
    \begin{itemize}
    \item \lst{span p l = (takeWhile p l, dropWhile p l)}
    \item \evalsto{span (<0) [-3..3]}{([-3,-2,-1],[0,1,2,3])}
    \end{itemize}
  \item \lst{break :: (a -> Bool) -> [a] -> ([a],[a])}
    \begin{itemize}
    \item \lst{break p l = (takeWhile q l, dropWhile q l)}\\
      \hspace{15ex}\lst{where q x = not (p x)}
    \item \evalsto{break (>0) [-3..3]}{([-3,-2,-1,0],[1,2,3])}
    \end{itemize}
  \end{itemize}
\end{frame}

\begin{frame}
  \frametitle{Логически квантори (\tt{any}, \tt{all})}

  \begin{itemize}[<+->]
  \item \lst{any :: (a -> Bool) -> [a] -> Bool}
    \begin{itemize}
    \item проверява дали предикатът е изпълнен за \textbf{някой елемент} от списъка
    \item \lst{any p l = or (map p l)}
    \item \lst{elem x = any (==x)}
    \end{itemize}
  \item \lst{all :: (a -> Bool) -> [a] -> Bool}
    \begin{itemize}
    \item проверява дали предикатът е изпълнен за \textbf{всички елементи} на списъка
    \item \lst{all p l = and (map p l)}
    \item \lst{sorted l = all (\\(x,y) -> x <= y) (zip l (tail l))}
    \end{itemize}
  \end{itemize}
\end{frame}

\end{document}
