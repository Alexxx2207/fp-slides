\documentclass{beamer}
\usepackage{fprog}

\title{Структури от данни в Scheme}
\subtitle{асоциативни списъи, матрици, дървета, графи}

\date{11 ноември 2015 г.}

\begin{document}

\begin{frame}
  \titlepage
\end{frame}

\section{Абстракция със структури от данни}

\subsection{Нива на абстракция}

\begin{frame}[<+->]
  \frametitle{Абстракция със структури от данни}
  
  \begin{definition}[Абстракция]
    Принцип за разделянето (``абстрахирането'') на \emph{представянето} на дадена структура от данни(СД) от нейното \emph{използване}. 
  \end{definition}
  \begin{itemize}
  \item основен принцип на обектно-ориентираното програмиране
  \item позволява използването на СД преди представянето ѝ да е уточнено
  \item Предимства:
    \begin{itemize}
    \item програмите работят на по-високо концептуално ниво със СД
    \item позволява алтернативни имплементации на дадена СД, подходящи за различен вид задача
    \item влиянието на промени по представянето е ограничено до операциите, които ``знаят'' за него
    \item подобрения при представянето автоматично се разпространяват до по-горните нива на абстракция
    \end{itemize}
  \end{itemize}
\end{frame}

\begin{frame}[<+->]
  \frametitle{Пример: рационално число}

  \begin{itemize}
  \item Логическо описание: обикновена дроб
  \item Физическо представяне: точкова двойка от цели числа
  \item Базови операции:
    \begin{itemize}[<.->]
    \item конструиране на рационално число
    \item получаване на числител
    \item получаване на знаменател
    \end{itemize}
  \item Аритметични операции:
    \begin{itemize}[<.->]
    \item събиране
    \item изваждане
    \item умножение
    \item деление
    \end{itemize}
  \item Приложни програми
  \end{itemize}
\end{frame}


\begin{frame}
  \frametitle{Нива на абстракция}

  \renewcommand{\bua}{\bigg\uparrow}

  \begin{center}
    \begin{tabular}{|c|}
      \hline
      приложни програми\\
      \\
      \hline
      \multicolumn 1c\bua\\
      \hline
      аритметични операции\\
      (+, -, *, /, =, <, >, <=, >=)\\
      \hline
      \multicolumn 1c\bua\\
      \hline
      базови операции\\
      (конструктор, селектори за числител и знаменател)\\
      \hline
      \multicolumn 1c\bua\\
      \hline
      физическо представяне\\
      (точкова двойка от цели числа)\\
      \hline
    \end{tabular}
  \end{center}
\end{frame}

\subsection{Абстракция с точкови двойки}

\begin{frame}[fragile]
  \frametitle{Рационални числа}
  
  Физическо представяне\\[1em]
  \begin{center}
    \begin{tabular}{cc}
      \hline
      \pointcell\\
      \hline
      \bda&\bda\\
      \fbox{числител} &\fbox{знаменател}
    \end{tabular}\\[1em]
  \end{center}
  \pause
  Базови операции
  \begin{itemize}[<+->]
  \item \alt<+->{\tt{(define make-rat cons)}}{\tt{(define (make-rat n d) (cons n d))}}
  \item \alt<+->{\tt{(define get-numer car)}}{\tt{(define (get-numer r) (car r))}}
  \item \alt<+->{\tt{(define get-denom cdr)}}{\tt{(define (get-denom r) (cdr r))}}
  \end{itemize}
  \onslide<+->
  \vspace{1em}
  По-добре:
\begin{verbatim}
(define (make-rat n d)
   (if (= d 0) (cons n 1) (cons n d)))
\end{verbatim}
\end{frame}

\begin{frame}[fragile]
  \frametitle{Аритметични операции}
  \small
  \begin{columns}[t,onlytextwidth]
    \begin{column}{0.33\textwidth}
      \vspace{2em}
      \begin{equation*}
        \frac{n_1}{d_1}\frac{n_2}{d_2} = \frac{n_1n_2}{d_1d_2}
      \end{equation*}
      \vspace{4em}
      \begin{uncoverenv}<2->
        \begin{equation*}
          \frac{n_1}{d_1} + \frac{n_2}{d_2} = \frac{n_1d_2 + n_2d_1}{d_1d_2}
        \end{equation*}
      \end{uncoverenv}
      \vspace{4em}
      \begin{uncoverenv}<3->
        \begin{equation*}
          \frac{n_1}{d_1} < \frac{n_2}{d_2} \leftrightarrow n_1d_2 < n_2d_1
        \end{equation*}
      \end{uncoverenv}
    \end{column}

    \begin{column}{0.67\textwidth}
\begin{verbatim}
(define (*rat p q)
  (make-rat (* (get-numer p) (get-numer q))
            (* (get-denom p) (get-denom q))))
\end{verbatim}
      \begin{uncoverenv}<2->
\begin{verbatim}
(define (+rat p q)
  (make-rat (+ (* (get-numer p)
                  (get-denom q))
               (* (get-denom p)
                  (get-numer q)))
            (* (get-denom p) (get-denom q))
\end{verbatim}
      \end{uncoverenv}
      \begin{uncoverenv}<3->
\begin{verbatim}
(define (<rat p q)
  (< (* (get-numer p) (get-denom q))
     (* (get-numer q) (get-denom p))))
\end{verbatim}
      \end{uncoverenv}
    \end{column}
  \end{columns}
\end{frame}

\begin{frame}[fragile]
  \frametitle{Програми с рационални числа}
  
  \begin{equation*}
    \sum_{i=0}^n \frac{i^n}{i!}
  \end{equation*}

  \onslide<+->

  \begin{overprint}
\begin{semiverbatim}
(define (my-exp x n)
  (accumulate \rvl{+rat} \rvl{(make-rat 0 1)} 0 n
              \rvl{(lambda (i) (make-rat (expt x i) (fact i)))} 1+))
\end{semiverbatim}
  \end{overprint}
\end{frame}

\begin{frame}[fragile]
  \frametitle{Нормализация}
  
  \textbf{Проблем:} Числата стават много големи!\\[1em]
  \textbf{Проблем:} \evalsto{(=rat (make-rat 1 2) (make-rat -1 -2))}{\#f}\\[1em]
  \pause
  \textbf{Идея:} Да работим с \emph{нормализирани} дроби
  \pause
\begin{verbatim}
(define (make-rat n d)
  (if (or (= d 0) (= n 0)) (cons 0 1)
    (let* ((g (gcd n d))
           (ng (quotient n g))
           (dg (quotient d g)))
       (if (> dg 0) (cons ng dg)
                    (cons (- ng) (- dg))))))
\end{verbatim}
  \pause
  \alert{Не е нужно да правим каквито и да е други промени!}
\end{frame}

\subsection{Абстракция със сигнатура}

\begin{frame}[fragile]
  \frametitle{Сигнатура}
  
  \textbf{Проблем:} Не можем да различим СД с еднакви представяния (рационално число, комплексно число, точка в равнината)\\[1em]
  \pause
  \textbf{Идея:} Да добавим ``етикет'' на обекта
  \pause
\begin{semiverbatim}
(define (make-rat n d)
  \alert{(cons 'rat}
    (if (or (= d 0) (= n 0)) (cons 0 1)
      (let* ((g (gcd n d))
             (ng (quotient n g))
             (dg (quotient d g)))
         (if (> dg 0) (cons ng dg)
                      (cons (- ng) (- dg)))))))

(define get-numer cadr)
(define get-denom cddr)
\end{semiverbatim}
\end{frame}

\begin{frame}[fragile,fragile]
  \frametitle{Проверка за коректност}
  
  Вече можем да проверим дали даден обект е рационално число:
\begin{verbatim}
(define (rat? p)
  (and (pair? p) (eq? (car p) 'rat)
       (pair? (cdr p)) (number? (cadr p)) (number? (cddr p))))
\end{verbatim}

  \pause

  Можем да добавим проверка за коретност:
\begin{verbatim}
(define (check-rat f)
  (lambda (p)
    (if (rat? p) (f p) 'error)))

(define get-numer (guard-rat cadr))
(define get-denom (guard-rat cddr))
\end{verbatim}
\end{frame}

\section{Матрици}

\subsection{Представяне}

\subsection{Операции}

\section{Двоични дървета}

\subsection{Представяне}

\subsection{Операции}

\section{Асоциативни списъци}

\subsection{Дефиниция}

\subsection{Операции}

\section{Графи}

\subsection{Представяне}

\subsection{Операции}

\subsection{Търсене в дълбочина}

\subsection{Търсене в ширина}

\end{document}
