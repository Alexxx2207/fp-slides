\documentclass[alsotrans]{beamerswitch}
\usepackage{fprog}

\title{Входно-изходни операции в Haskell}

\date{15 януари 2020 г.}

\lstset{language=Haskell,style=Haskell}

\begin{document}

\begin{frame}
  \titlepage
\end{frame}

\section{Странични ефекти в Haskell}

\begin{frame}
  \frametitle{Странични ефекти в Haskell}

  \begin{itemize}[<+->]
  \item Функциите в Haskell нямат странични ефекти
  \item Но входно-изходните операции по природа са странични ефекти!
  \item Как можем да се справим с този парадокс?
  \item \textbf{Идея:} Можем да си мислим за входно-изходните операции като поточна обработка на данни
  \end{itemize}
  \onslide<+->
  \begin{center}
    производител
    $\longrightarrow$
    \begin{tikzpicture}[scale=.7,baseline={(0,0)}]
      \filldraw[draw=black,fill=diagramblue]
      (0,1) .. controls +(2,1.5) and +(-2,-1.5) .. (8,1) --
      (8,-1) .. controls +(-2,-1.5) and +(2,1.5) .. (0,-1) -- (0,1);
      \path (0,0) .. controls +(2,1.5) and +(-2,-1.5) .. (8,0)
      \foreach \p in {.1,.2,...,.9} {
        node[circle,inner sep=0,minimum size=6,draw=black,fill=yellow!40!white,pos=\p] {}
      };
    \end{tikzpicture} $\longrightarrow$ консуматор
  \end{center}
\end{frame}

\begin{frame}
  \frametitle{Поточна обработка}
  \textbf{Задача.} Да се въведат \tt{n} числа и да се изведе тяхното средно аритметично.\\
  \pause
  \textbf{Решение:} Дефинираме трансформация над стандартните вход и изход, която:
  \begin{itemize}[<+->]
  \item приема \tt{n} като параметър
  \item трансформира входния поток, като \alert{консумира} от него \tt{n} числа, записвайки ги в списък
  \item пресмята средното аритметично \tt{avg} на числата в списъка
  \item трансформира изходния поток, като \alert{произвежда} върху него низовото представяне на \tt{avg}
  \end{itemize}
  \onslide<+->
  \textbf{Трансформирането} на входно-изходните потоци несъмнено е страничен ефект, но \textbf{конструирането на трансформацията} няма нужда от странични ефекти!\\
  \onslide<+->
  \alert{Функциите, които работят с вход и изход, по същество дефинират композиция на входно-изходни трансформации.}
\end{frame}

\section{\tt{IO}}

\begin{frame}
  \frametitle{Типът \tt{IO a}}

  Стандартният генеричен тип \lst{IO a} задава тип на входно/изходна трансформация, резултатът от която е от тип \tt{a}.\\
  \pause
  \textbf{Частен случай:} \lst{IO ()} задава трансформация, която връща празен резултат.\\[1em]
  \pause
  \textbf{Входни трансформации:}
  \begin{itemize}
  \item \lst{getChar :: IO Char} --- връща символ, прочетен от входа
  \item \lst{getLine :: IO String} --- връща ред, прочетен от входа
  \end{itemize}
  \pause
  \textbf{Изходни трансформации:}
  \begin{itemize}
  \item \lst{putChar :: Char -> IO ()} --- извежда символ на изхода
  \item \lst{putStr :: String -> IO ()} --- извежда низ на изхода
  \item \lst{putStrLn :: String -> IO ()} --- извежда ред на изхода
  \end{itemize}
\end{frame}

\begin{frame}[fragile]
  \frametitle{Главна функция \tt{main}}

  \begin{itemize}[<+->]
  \item Функцията \lst{main :: IO ()} от модула \lst{Main} в Haskell е
    специална: тя е входната точка на компилираната програма.
  \item По същество тя дефинира входно-изходна трансформация, която се прилага към стандартния вход и изход при изпълнение на програмата.
  \item \textbf{Пример:} \lst^main = putStrLn "Hello, world!"^
  \item Можем ли да дефинираме \lst!main = putStrLn $ "Въведохте: " ++ getLine!? % поправка на оцветяването на синтаксис$
  \item \alert{Не!} \hspace{5ex} \lst{getLine :: }\tta{IO} \lst{String}
  \item Композицията на входно-изходни трансформации не работи като композицията на функции
  \item Низът, който връща \lst{getLine} е ``замърсен'' от входно-изходна операция
  \item Как да композираме трансформации?
  \end{itemize}
\end{frame}

\section{Синтаксис за вход/изход}

\begin{frame}[fragile]
  \frametitle{Конструкцията \lst{do}}

  В Haskell има специален синтаксис за композиране на трансформации:\\[1em]
  \tta{do} \{ <трансформация> \}\\[1em]
  \pause
  <трансформация> може да бъде:
  \begin{itemize}[<+->]
  \item произволен израз от тип \lst{IO a}
  \item{} <име> \tta{<-} <трансформация>
    \begin{itemize}
    \item{} <трансформация> е от тип \lst{IO a}
    \item резултатът от <трансформация> се свързва с <име>
    \end{itemize}
  \item \tta{return} <израз>
    \begin{itemize}
    \item празна трансформация, която връща <израз> като резултат
    \item \lst{return :: a -> IO a}
    \end{itemize}
  \item резултатът от цялата конструкция \tt{do} е резултатът от последната трансформация в композицията
  \end{itemize}
  \onslide<+->
\begin{lstlisting}
main = do line <- getLine
          putStrLn $ "Въведохте: " ++ line
\end{lstlisting}
\end{frame}

\begin{frame}[fragile,fragile]
  \frametitle{Локални дефиниции в \lst{do}}
  \small
  В някакъв смисъл \lst{<-} и \lst{return} са обратни една на друга операции:
  \begin{itemize}[<+->]
  \item \lst{<-} извлича ``чист'' резултат от тип \tt{a} от трансформация от тип \lst{IO a}
  \item \lst{return} фиктивно ``замърсява'' резултат от тип \tt{a} за да стане от тип \lst{IO a}
  \item Какъв е ефектът от <име> \lst{<- return} <израз> в \lst{do} конструкция?
  \item Създава се локалната дефиниция <име> = <израз>!
  \item Понеже се налага често, има специален синтаксис за това:\\
    \lst{let} <име> \tta= <израз>
  \item Да не се бърка с \lst{let} <име> \tt= <израз> \tta{in} <израз>!
  \end{itemize}
  \onslide<+->
  \textbf{Пример:}
  \vspace{-.5ex}
\begin{lstlisting}
main = do putStrLn "Моля, въведете палиндром: "
          line <- getLine
          let revLine = reverse line
          if revLine == line then putStrLn "Благодаря!"
          else do putStrLn $ line ++ " не е палиндром!"
                  main
\end{lstlisting}
\end{frame}

\begin{frame}
  \frametitle{Вход и изход на данни}

  Как можем да извеждаме и въвеждаме данни от типове различни от \lst{Char} и \tt{String}?\\[1em]
  \pause
  На помощ идват класовете \lst{Show} и \lst{Read}:
  \begin{itemize}[<+->]
  \item \lst{show :: (Show a) => a -> String}
  \item \lst{print :: (Show a) => a -> IO ()}
  \item \lst{print = putStrLn . show}
  \item \lst{read :: (Read a) => String -> a}
  \item \evalstoerrp{read "1.23"}
  \item Haskell не може да познае типа на резултата, понеже е генеричен!
  \item \lst{getInt :: IO Int}
  \item \lst{getInt = do line <- getLine}\\
    \hspace{15ex}\lst{return \$ read line}
  \end{itemize}
\end{frame}

\begin{frame}[fragile]
  \frametitle{Пример: средно аритметично на редица от числа}
\begin{lstlisting}
findAverage :: IO Double@\pause@
findAverage = do putStr "Моля, въведете брой: "
                 n <- getInt
                 s <- readAndSum n
                 return $ fromIntegral s / fromIntegral n
@\pause@
readAndSum :: Int -> IO Int@\pause@
readAndSum 0 = return 0
readAndSum n = do putStr "Моля, въведете число: "
                  x <- getInt
                  s <- readAndSum $ n - 1
                  return $ x + s
@\pause@
main = do avg <- findAverage
          putStrLn $ "Средното аритметично е: " ++ show avg
\end{lstlisting}
\end{frame}

\section{Управляващи структури}


\begin{frame}
  \frametitle{Управляващи функции}

  Можем да работим с трансформации с функции от по-висок ред:
  \begin{itemize}[<+->]
  \item \lst{import Control.Monad}
  \item \lst{sequence :: [IO a] -> IO [a]}
    \begin{itemize}
    \item композира трансформации и събира резултатите им в списък
    \item \alt<+->{\lst!getInts = sequence . (`replicate` getInt)!}{\lst!getInts n = sequence $ replicate n getInt!} % поправка на оцветяването $
    \end{itemize}
  \item \lst{mapM :: (a -> IO b) -> [a] -> IO [b]}
    \begin{itemize}
    \item композира списък от трансформации по списък от стойности
    \item \lst{mapM = sequence . map}
    \item \lst!printRead s = do putStr $ s ++ " = "; getInt! % поправка на оцветяването $
    \item \lst{readCoordinates = mapM printRead ["x", "y", "z"]}
    \end{itemize}
  \item \lst{mapM_ :: (a -> IO b) -> [a] -> IO ()}
    \begin{itemize}
    \item Също като \lst{mapM}, но изхвърля резултата
    \item \lst{printList = mapM_ print}
    \end{itemize}
  \item \lst{forever :: IO a -> IO b}
    \begin{itemize}
    \item безкрайна композиция на една и съща трансформация
    \item както \lst{repeat} за списъци
    \item \lst{forever \$ do line <- getLine; putStrLn line}
    \end{itemize}
  \end{itemize}
\end{frame}

\begin{frame}[fragile]
  \frametitle{Средно аритметично на числа v2.0}

  \small
\begin{lstlisting}
readInt :: String -> IO Int
readInt s = do putStr $ "Моля, въведете " ++ s ++ ": "
               getInt

findAverage :: IO Double
findAverage = do n <- readInt "брой"
                 l <- mapM (readInt.("число #"++).show) [1..n]
                 let s = sum l
                 return $ fromIntegral s / fromIntegral n

main = forever $
       do avg <- findAverage
          putStrLn $ "Средното аритметично е: " ++ show avg
          putStrLn "Хайде отново!"
\end{lstlisting}
\end{frame}

\begin{frame}[fragile]
  \frametitle{Ленив вход и изход}

  \begin{itemize}[<+->]
  \item Ленивото оценяване в Haskell ни позволява да работим с входно/изходни потоци
  \item \lst{getContents :: IO String} --- връща списък от \alert{всички} символи на стандартния вход
  \item списъкът се оценява лениво, т.е. прочита се при нужда
  \item \textbf{Пример:}
\begin{lstlisting}
noSpaces = do text <- getContents
              putStr $ filter (/=' ') text
\end{lstlisting}
  \item \lst{interact :: (String -> String) -> IO ()} --- лениво прилага функция над низове над стандартния вход и извежда резултата на стандартния изход
  \item \textbf{Пример:}
\begin{lstlisting}
noSpaces = interact $ filter (/=' ')
\end{lstlisting}
  \end{itemize}
\end{frame}

\begin{frame}[fragile]
  \frametitle{Работа с файлове}

  \begin{itemize}[<+->]
  \item \lst{IO} позволява работа с произволни файлове, не само със стандартните вход и изход
  \item \lst{import System.IO}
  \item \lst{openFile :: FilePath -> IOMode -> IO Handle} --- отваря файл със зададено име в зададен режим
    \begin{itemize}
      \footnotesize
    \item \lst{type FilePath = String}
    \item \lst{data IOMode = ReadMode|WriteMode|AppendMode|ReadWriteMode}
    \end{itemize}
  \item Има варианти на функциите за вход/изход, които работят с \lst{Handle}
  \item \lst{hGetLine}, \lst{hGetChar}, \lst{hPutStr}, \lst{hPutStrLn}, \lst{hGetContents}\ldots
  \item \textbf{Пример:}
    \vspace{-1ex}
\begin{lstlisting}
encrypt cypher inFile outFile =
  do h1 <- openFile inFile ReadMode
     text <- hGetContents h1
     h2 <- openFile outFile WriteMode
     hPutStr h2 $ map cypher text
\end{lstlisting}
  \end{itemize}
\end{frame}

\section{Монади}

\begin{frame}
  \frametitle{Монади}

  \begin{itemize}[<+->]
  \item \lst{IO} е пример за \alert{монада}
  \item Монадите са конструкции, които ``опаковат'' обекти от даден тип
  \item \textbf{Примери:}
    \begin{itemize}
    \item \lst{IO} опакова стойност във входно/изходна трансформация
    \item \lst{Maybe} опакова стойност с ``флаг'' дали стойността съществува
    \item \lst{[a]} опакова няколко ``алтернативни'' стойности в едно
    \item \lst{r -> a} опакова стойност от тип \lst{a} в ``машинка'', която я пресмята при подаден параметър от тип \lst{r}
    \item \lst{s -> (a,s)} опакова стойност от тип \lst{a} в ``действие'', което променя дадено състояние от тип \lst{s}
    \end{itemize}
  \end{itemize}
\end{frame}

\begin{frame}
  \frametitle{Монадни операции}

  \begin{itemize}[<+->]
  \item \lst{Monad} е клас от \alert{типови конструктори}, които
    са монади
  \item ``Опаковката'' често е \alert{еднопосочна}: един път опакована, не можем да извадим стойността извън опаковката\ldots
  \item \ldots но можем да я преопаковаме!
  \item \lst{(>>=) :: Monad m => m a -> (a -> m b) -> m b}
  \item оператор за ``свързване'' на опаковани стойности
  \item \lst{b = a >>= f}:
    \begin{itemize}
    \item поглеждаме стойността \lst{x} в опаковката \lst{a}
    \item прилагаме функцията \lst{f} над \lst{x}
    \item и получаваме нова опакована стойност \lst{b}
    \end{itemize}
  \end{itemize}
\end{frame}

\begin{frame}[fragile]
  \frametitle{Императивен стил чрез монади}

  \small
  \begin{fixedarea}
    \begin{itemize}[<+->]
    \item \lst{do} всъщност е синтактична захар за поредица от ``свързвания``
    \item \textbf{Примери:}
      \begin{onlyenv}<2>
\begin{lstlisting}
main = do line <- getLine
          putStrLn $ "Въведохте: " ++ line
\end{lstlisting}
      \end{onlyenv}
      \begin{onlyenv}<3>
\begin{lstlisting}
main = getLine >>= (\line -> putStrLn $ "Въведохте " ++ line)
\end{lstlisting}
      \end{onlyenv}
      \begin{onlyenv}<4->
\begin{lstlisting}
main = getLine >>= putStrLn . ("Въведохте: " ++)
\end{lstlisting}
      \end{onlyenv}
      \begin{onlyenv}<5>
\begin{lstlisting}
findAverage = do putStr "Моля, въведете брой: "
                 n <- getInt
                 s <- readAndSum n
                 return $ fromIntegral s / fromIntegral n
\end{lstlisting}
      \end{onlyenv}
      \begin{onlyenv}<6->
\begin{lstlisting}
findAverage = putStr "Моля, въведете брой: " >>=
               (\_ -> getInt >>=
               (\n -> readAndSum n >>=
               (\s -> return $ fromIntegral s /
                               fromIntegral n))
\end{lstlisting}
      \end{onlyenv}
    \item<7-> работи за произволни монади, не само за \lst{IO}!
    \item<8-> позволява абстрахиране от страничните ефекти и моделиране на поредица от инструкции
    \item<9-> императивен стил във функционалното програмиране
    \end{itemize}
  \end{fixedarea}
\end{frame}
\end{document}
