\documentclass{beamer}
\usepackage{fprog}

\title{Списъци}

\date{4 ноември 2015 г.}

\begin{document}

\begin{frame}
  \titlepage
\end{frame}

\section{Точкови двойки}

\begin{frame}
  \frametitle{Точкови двойки}

  \begin{columns}[t,onlytextwidth]
    \column{0.5\textwidth}
    \vspace{1em}

    \centering
    \tt{(A . B)}

    \column{0.5\textwidth}

    \centering
    \begin{tabular}{cc}
      \hline
      \pointcell\\
      \hline
      \bda&\bda\\
      \fbox A &\fbox B
    \end{tabular}
  \end{columns}

  \pause
  \vspace{1em}

  \begin{itemize}[<+->]
  \item \tta{(cons }<израз$_1$> <израз$_2$>\tta)
  \item Точкова двойка от оценките на <израз$_1$> и <израз$_2$>
  \item \tta{(car }<израз>\tta)
  \item \textbf{Първият} компонент на двойката, която е оценката на <израз>
  \item \tta{(cdr }<израз>\tta)
  \item \textbf{Вторият} компонент на двойката, която е оценката на <израз>
  \item \tta{(pair? }<израз>\tta)
  \item Проверява дали оценката на <израз> е точкова двойка
  \end{itemize}

\end{frame}

\begin{frame}[fragile]
  \frametitle{Примери}

  \begin{columns}[t,onlytextwidth]

    \column{0.6\textwidth}

    \begin{tabular}{c}
      \tt{(cons (cons 2 3) (cons 8 13))}\\
      \bda\\
      \tt{((2 . 3) . (8 . 13))}
    \end{tabular}
    \vspace{4em}

\onslide<3->{
  \begin{tabular}{c}
    \tt{(cons 3 (cons (cons 13 21) 8))}\\
    \bda\\
    \tt{(3 . ((13 . 21) . 8))}
  \end{tabular}
}
    \column{0.4\textwidth}

\onslide<2->{
    \begin{tabular}{cc@{}c@{}cc}
      \cline{1-2}\cline{4-5}
      \pointcell&$\nextarrow$&\pointcell\\
      \cline{1-2}\cline{4-5}
      \bda&&&\bda&\bda\\
      \cline{1-2}
      \pointcell&&\fbox8&\fbox{13}\\
      \cline{1-2}
      \bda&\bda\\
      \fbox2 &\fbox 3
    \end{tabular}
    \vspace{1em}
}

\onslide<4>{
      \begin{tabular}{ccc@{}c@{}c}
      \cline{1-2}
      \pointcell\\
      \cline{1-2}
      \bda&\bda\\
      \cline{2-3}
      \fbox3&\pointcell&$\nextarrow$&\fbox8\\
      \cline{2-3}
      &\bda\\
      \cline{2-3}
      &\pointcell\\
      \cline{2-3}
      &\bda&\bda\\
      &\fbox{13}&\fbox{21}
    \end{tabular}
}
  \end{columns}
\end{frame}

\begin{frame}
  \frametitle{S-изрази}

  \begin{definition}
    S-израз наричаме:
    \begin{itemize}
    \item атоми (булеви, числа, знаци, символи, низове)
    \item точкови двойки \tt{(S$_1$ . S$_2$)}, където \tt{S$_1$} и \tt{S$_2$} са S-изрази
    \end{itemize}
  \end{definition}
  \vspace{1em}

  \pause

  \alert{S-изразите са най-общия тип данни в Scheme.}

  \vspace{1em}
  С тяхна помощ могат да се дефинират произволно сложни структури от данни.
\end{frame}

\begin{frame}[fragile]
  \frametitle{All you need is $\lambda$ --- точкови двойки}

  Можем да симулираме \tt{cons}, \tt{car} и \tt{cdr} чрез \tt{lambda}!
  \vspace{1em}

  \pause

  \textbf{Вариант №1:}

\begin{verbatim}
(define (lcons x y) (lambda (p) (if p x y)))
(define (lcar z) (z #t))
(define (lcdr z) (z #f))
\end{verbatim}

  \pause

  \textbf{Вариант №2:}

\begin{verbatim}
(define (lcons x y) (lambda (p) (p x y)))
(define (lcar z) (z (lambda (x y) x)))
(define (lcdr z) (z (lambda (x y) y)))
\end{verbatim}
\end{frame}

\section{Списъци}

\begin{frame}
  \frametitle{Списъци в Scheme}

  \begin{definition}
    \begin{enumerate}
    \item Празният списък \tt{()} е списък
    \item \tt{(h . t)} е списък ако \tt t е списък
      \begin{itemize}
      \item \tt h --- глава на списъка
      \item \tt t --- опашка на списъка
      \end{itemize}
    \end{enumerate}
  \end{definition}

  \pause
  \vspace{2em}

  \begin{tabular}{*2{cc@{}c@{}}@{\hskip 1ex}cc@{}cc}
    \cline{1-2}\cline{4-5}\cline{9-10}
    \nextcell&\nextcell&\ldots&$\nextarrow$&\nilcell\\
    \cline{1-2}\cline{4-5}\cline{9-10}
    \bda&&&\bda&&&&&\bda\\
    \fbox{$a_1$}&&&\fbox{$a_2$}&&&&&\fbox{$a_n$}
  \end{tabular}

  \pause
  \vspace{2em}

  \tt{($a_1$ . ($a_2$ . ( \ldots ( $a_n$ . () ) ) ) ) $\longleftrightarrow$ ($a_1$ $a_2$ \ldots\ $a_n$)}
\end{frame}

\begin{frame}
  \frametitle{Вградени функции за списъци}

  \begin{itemize}[<+->]
  \item \tta{(null? }<израз>\tta) --- дали <израз> е празният списък \tt{()}
  \item \tta{(list? }<израз>\tta) --- дали <израз> е списък
    \begin{itemize}
    \item \scriptsize \tt{(define (list? l) (or (null? l) (and (pair? l) (list? (cdr l)))))}
    \end{itemize}
  \item \tta{(list }\{<израз>\}\tta) --- построява списък с елементи <израз>
  \item \tt{(list }<израз$_1$> <израз$_2$> \ldots <израз$_n$>\tt) $\longleftrightarrow$\\
    \tt{(cons }<израз$_1$>\tt{ (cons }<израз$_2$> \ldots\tt{ (cons }<израз$_n$>\tt{ '()))))}
  \item \tt{(cons }<глава> <опашка>\tt) --- списък с <глава> и <опашка>
  \item \tt{(car }<списък>\tt) --- главата на <списък>
  \item \tt{(cdr }<списък>\tt) --- опашката на <списък>
  \item \alert{\tt{()} не е точкова двойка!}
  \item \evalstoerr{(car '())}, \evalstoerr{(cdr '())}
  \end{itemize}
\end{frame}

\begin{frame}
  \frametitle{Съкратени форми на \tt{car} и \tt{cdr}}

  Нека $l = (a_1\,a_2\,a_3\,\ldots\,a_n)$.

  \begin{itemize}[<+->]
  \item \evalsto{(car l)}{$a_1$}
  \item \evalsto{(cdr l)}{$(a_2\,a_3\,\ldots\,a_n)$}
  \item \evalstwop{(car (cdr l))}{$a_2$}{(cadr l)}
  \item \evalstwop{(cdr (cdr l))}{$(a_3\,\ldots\,a_n)$}{(cddr l)}
  \item \evalstwop{(car (cdr (cdr l)))}{$a_3$}{(caddr l)}
  \item имаме съкратени форми за до 4 последователни прилагания на \tt{car} и \tt{cdr}
  \end{itemize}
\end{frame}

\begin{frame}
  \frametitle{Форми на равенство в Scheme}

  \begin{itemize}[<+->]
  \item \tta{(eq? }<израз$_1$> <израз$_2$>\tta) --- връща \tt{\#t} точно тогава, когато оценките на <израз$_1$> <израз$_2$> заемат едно и също място в паметта
  \item \tta{(eqv? }<израз$_1$> <израз$_2$>\tta) --- връща \tt{\#t} точно тогава, когато оценките на <израз$_1$> и <израз$_2$> заемат едно и също място в паметта или са едни и същи по стойност \textbf{атоми} (дори и да заемат различно място в паметта)
    \begin{itemize}
    \item Ако \tt{(eq? }<израз$_1$> <израз$_2$>\tt),\\
      то със сигурност \tt{(eqv? }<израз$_1$> <израз$_2$>\tt)
    \end{itemize}
  \item \tta{(equal? }<израз$_1$> <израз$_2$>\tta) --- връща \tt{\#t} точно тогава, когато оценките на <израз$_1$> и <израз$_2$> са едни и същи по стойност \textbf{атоми или точкови двойки}, чиито компоненти са равни в смисъла на \tt{equal?}
    \begin{itemize}
    \item В частност, \tt{equal?} проверява за равенство на списъци
    \item Ако \tt{(eqv? }<израз$_1$> <израз$_2$>\tt),\\
      то със сигурност \tt{(equal? }<израз$_1$> <израз$_2$>\tt)
    \end{itemize}
  \end{itemize}
\end{frame}

\begin{frame}
  \frametitle{Вградени функции за списъци}
  \small

  \begin{itemize}[<+->]
  \item \tta{(length }<списък>\tta) --- връща дължината на <списък>
  \item \tta{(append }\{<списък>\}\tta) --- конкатенира всички <списък>
  \item \tta{(reverse }<списък>\tta) --- елементите на <списък> в обратен ред
  \item \tta{(list-tail }<списък> n\tta) --- елементите на <списък> без първите n
  \item \tta{(list-ref }<списък> n\tta) --- n-ти елемент на <списък> (от 0)
  \item \tta{(member }<елемент> <списък>\tta) --- проверява дали <елемент> се среща в <списък>
    \begin{itemize}
    \item По-точно, връща <списък> от <елемент> нататък, ако го има
    \item Връща \tt{\#f}, ако <елемент> го няма в <списък>
    \item Сравнението на елементи става с \tt{equal?}
    \end{itemize}
  \item \tta{(memqv }<елемент> <списък>\tta) --- като \tt{member}, но сравнява с \tt{eqv?}
  \item \tta{(memq }<елемент> <списък>\tta) --- като \tt{member}, но сравнява с \tt{eq?}
  \end{itemize}
\end{frame}

\subsection{Рекурсия над списъци}


\begin{frame}
  \frametitle{Обхождане на списъци}

  При обхождане на \tt l:
  \begin{itemize}
  \item Ако \tt l е празен, връщаме базова стойност \textbf{(дъно)}
  \item Иначе, комбинираме главата \tt{(car l)} с резултата от рекурсивното извикване над опашката \tt{(cdr l)} \textbf{(стъпка)}
  \end{itemize}

  \pause
  \vspace{2em}
  Примери: \tt{length}, \tt{list-tail}, \tt{list-ref}, \tt{member}, \tt{memqv}, \tt{memq}
\end{frame}

\begin{frame}
  \frametitle{Конструиране на списъци}

  Използваме рекурсия по даден параметър (напр. число, списък...)
  \begin{itemize}
  \item На дъното връщаме фиксиран списък (например \tt{()})
  \item На стъпката построяваме с \tt{cons} списък със съответната глава, а опашката строим чрез рекурсивно извикване на същата функция
  \end{itemize}

  \pause
  \vspace{2em}
  Примери: \tt{from-to}, \tt{collect}, \tt{append}, \tt{reverse}
\end{frame}

\section{Функции от по-висок ред за списъци}

\begin{frame}[fragile]
  \frametitle{Изобразяване на списък (\tt{map})}

  % TODO картинка
  Да се дефинира функция \tt{(map <функция> <списък>)}, която връща нов списък съставен от елементите на <списък>, върху всеки от които е приложена <функция>.

  \pause

\begin{verbatim}
(define (map f l)
        (if (null? l) '()
            (cons (f (car l)) (map f (cdr l)))))
\end{verbatim}

  \pause

  Примери:
  \small
  \begin{itemize}[<+->]
  \item \evalstop{(map square '(1 2 3))}{(1 4 9)}
  \item \evalstop{(map cadr '((a b c) (d e f) (g h i)))}{(b e h)}
  \item \evalstop{(map (lambda (f) (f 2)) (list square 1+ odd?))}{(4 3 \#f)}
  \item \evalstop{(map (lambda (f) (f 2)) (map twice (list square 1+ boolean?)))}{(16 4 \#t)}
  \end{itemize}
\end{frame}

\begin{frame}[fragile]
  \frametitle{Филтриране на списък (\tt{filter})}

  Да се напише функция \tt{(filter }<условие> <списък>\tt), която връща само тези от елементите на <списък>, които удовлетворяват <условие>.

\pause
\small

\begin{verbatim}
(define (filter p? l)
   (cond ((null? l) l)
         ((p? (car l)) (cons (car l) (filter p? (cdr l))))
         (else (filter p? (cdr l)))))
\end{verbatim}

\pause

  Примери:
  \begin{itemize}[<+->]
  \item \evalstop{(filter odd? '(1 2 3 4 5))}{(1 3 5)}
  \item \evalstop{(filter pair? '((a b) c () d (e)))}{((a b) (e))}
  \item \evalstop{(map (lambda (x) (filter even? x)) '((1 2 3) (4 5 6) (7 8 9)))}{((2) (4 6) (8))}
  \item \evalstop{(map (lambda (x) (map (lambda (f) (filter f x)) (list negative? zero? positive?))) '((-2 1 0) (1 4 -1) (0 0 1)))}{(((-2) (0) (1)) ((-1) () (1 4)) (() (0 0) (1)))}
  \end{itemize}

\end{frame}

\begin{frame}[fragile]
  \frametitle{Дясно свиване (\tt{foldr})}

  Да се напише функция, която по даден списък $l = (a_1\,a_2\,a_3\,\ldots\,a_n)$ пресмята:
  \begin{equation*}
    a_1 \oplus \Big(a_2 \oplus \big(\ldots \oplus (a_n \oplus \bot) \ldots\big)\Big),
  \end{equation*}

  \pause

\begin{verbatim}
(define (foldr op nv l)
        (if (null? l) nv
            (op (car l) (foldr op nv (cdr l)))))
\end{verbatim}

  \pause
  Примери:
  \small
  \begin{itemize}[<+->]
  \item \evalstop{(foldr * 1 (from-to 1 5))}{120}
  \item \evalstop{(foldr + 0 (map square (filter odd? (from-to 1 5))))}{35}
  \item \evalstop{(foldr cons '() '(1 5 10))}{(1 5 10)}
  \item \evalstop{(foldr list '() '(1 5 10))}{(1 (5 (10 ()))}
  \item \evalstop{(foldr append '() '((a b) (c d) (e f)))}{(a b c d e f)}
  \item \tt{map}, \tt{filter} и \tt{accumulate} могат да се реализират чрез \tt{foldr}
  \end{itemize}

\end{frame}

\begin{frame}[fragile]
  \frametitle{Ляво свиване (\tt{foldl})}

  Да се напише функция, която по даден списък $l = (a_1\,a_2\,a_3\,\ldots\,a_n)$ пресмята:
  \begin{equation*}
    \Big(\ldots\big((\bot \oplus a_1) \oplus a_2\big) \oplus \ldots\Big) \oplus a_n
  \end{equation*}

  \pause

\begin{verbatim}
(define (foldl op nv l)
  (if (null? l) nv
      (foldl op (op nv (car l)) (cdr l))))
\end{verbatim}

  \pause

  Примери:
  \small
  \begin{itemize}[<+->]
  \item \evalstop{(foldl * 1 (from-to 1 5)}{120}
  \item \evalstop{(foldl cons '() '(1 5 10))}{(((() . 1) . 5) . 10)}
  \item \evalsto{(foldl \rvl{(lambda (x y) (cons y x))} '() '(1 5 10))}{(10 5 1)}
  \item \evalstop{(foldl list '() '(1 5 10))}{(((() 1) 5) 10)}
  \item \evalstop{(foldl append '() '((a b) (c d) (e f)))}{(a b c d e f)}
  \item \tt{foldr} генерира линеен рекурсивен процес, а \tt{foldl} --- линеен итеративен
  \end{itemize}
\end{frame}

\begin{frame}[fragile]
  \frametitle{Функции от по-висок ред в Racket}

  В R$^5$RS е дефинирана само функцията \tt{map}.

  В Racket са дефинирани функциите \tt{map}, \tt{filter}, \tt{foldr}, \tt{foldl}

  \pause
  \vspace{1em}

  \alert{Внимание: \tt{foldl} в Racket е дефинирана по-различен начин!}

  \pause

  \begin{columns}[t,onlytextwidth]
    \small
    \column{0.5\textwidth}

    \tt{foldl} от лекции

\begin{semiverbatim}
(define (foldl op nv l)
  (if (null? l) nv
      (foldl op \alert{(op nv (car l))}
             (cdr l))))
\end{semiverbatim}

    \begin{equation*}
      \Big(\ldots\big((\bot \oplus a_1) \oplus a_2\big) \oplus \ldots\Big) \oplus a_n
    \end{equation*}

    \column{0.5\textwidth}

    \tt{foldl} в Racket

\begin{semiverbatim}
(define (foldl op nv l)
  (if (null? l) nv
      (foldl op \alert{(op (car l) nv)}
             (cdr l))))
\end{semiverbatim}

    \begin{equation*}
      a_n \oplus \Big(\ldots \big(a_2 \oplus (a_1 \oplus \bot)\big)\ldots\Big),
    \end{equation*}

  \end{columns}
\end{frame}

\begin{frame}
  \frametitle{Свиване на непразен списък (\tt{foldr1, foldl1})}

  \textbf{Задача.} Да се намери максималният елемент на списък.
  \pause

  \tt{(define (maximum l) (foldr max \alt<2>?{(car l)} l))}
  \pause\pause

  \vspace{1em}
  Можем ли да пропуснем нулевата стойност за непразен списък?

  \begin{columns}[t,onlytextwidth]
    \column{0.45\textwidth}

    \onslide<5->{
    \begin{equation*}
      a_1 \oplus \big(\ldots \oplus (a_{n-1} \oplus a_n) \ldots\big)
    \end{equation*}}

    \onslide<7->{
    \begin{equation*}
      \big(\ldots\big((a_1 \oplus a_2) \oplus \ldots\big) \oplus a_n
    \end{equation*}}

    \column{0.55\textwidth}

    \vspace{1em}
    \onslide<6->{
      \tt{(define (foldr1 op l)}\\
      \tt{\hspace{2ex}(if (null? (cdr l)) (car l)}\\
      \tt{\hspace{7ex}(op (car l)}\\
      \tt{\hspace{12ex}(foldr1 op (cdr l)))))}}

    \vspace{2em}

    \onslide<8->{
      \tt{(define (foldl1 op l)}\\
      \tt{\hspace{2ex}(foldl op (car l) (cdr l)))}}
  \end{columns}
\end{frame}
\end{document}
\section{Вложени списъци}

\newcommand{\hzero}{\textcolor{red}}
\newcommand{\vzero}{\textcolor{orange}}
\newcommand{\hstep}{\textcolor{green}}
\newcommand{\vstep}{\textcolor{cyan}}

\begin{frame}[fragile]
  \frametitle{Работа с вложени списъци}

\begin{verbatim}
((1 (2)) (((3) 4) (5 (6)) () (7)) 8)
\end{verbatim}

  \textbf{Задача.} Да се преброят в атомите във вложен списък.

  \pause

  \textbf{Подход:} Обхождане в две посоки: хоризонтално и вертикално
  \begin{itemize}[<+->]
  \item \hzero{Хоризонтално дъно:} \rvl{достигане до празен списък \tt{()}}
  \item \vzero{Вертикално дъно:} \rvl{достигане до друг атом}
  \item \hstep{Хоризонтална стъпка:} \rvl{обхождане опашката \tt{(cdr l)}}
  \item \vstep{Вертикална стъпка:} \rvl{обхождане на главата \tt{(car l)}}
  \end{itemize}
  \vspace{1em}

  \nxt{
  За удобство можем да дефинираме функцията \tt{atom?}:
\begin{semiverbatim}
(define (atom? x) (not (pair? x)))
\end{semiverbatim}}
\end{frame}

\begin{frame}[fragile]
  \frametitle{Примери}
  \small

  \textbf{Задача.} Да се преброят в атомите във вложен списък.

  \evalsto{(count-atoms '((1 (2)) (((3) 4) (5 (6)) () (7)) 8))}8

  \pause

\begin{semiverbatim}
(define (count-atoms l)
  (cond ((null? l) \hzero0)
        ((atom? l) \vzero1)
        (else (+ \vstep{(count-atoms (car l))} \hstep{(count-atoms (cdr l))}))))
\end{semiverbatim}
  \vspace{1em}

  \pause

  \textbf{Задача.} Да се съберат всички атоми от вложен списък.

  \evalsto{(flatten '((1 (2)) (((3) 4) (5 (6)) () (7)) 8))}{(1 2 3 4 5 6 7 8)}

  \pause

\begin{semiverbatim}
(define (flatten l)
  (cond ((null? l) \hzero{'()})
        ((atom? l) \vzero{(list l)})
        (else (append \vstep{(flatten (car l))} \hstep{(flatten (cdr l))}))))
\end{semiverbatim}

\end{frame}

\begin{frame}[fragile]
  \frametitle{Примери}

  \textbf{Задача.} Да се обърне реда на атомите във вложен списък.
  \evalsto{(deep-reverse '((1 (2)) (((3) 4) (5 (6)) () (7)) 8))}{\\
(8 ((7) () ((6) 5) (4 (3))) ((2) 1))}

  \pause

\begin{semiverbatim}
(define (deep-reverse l)
  (cond ((null? l) \hzero{'()})
        ((atom? l) \vzero{l})
        (else (append \hstep{(deep-reverse (cdr l))}
                \vstep{(list (deep-reverse (car l)))}))))
\end{semiverbatim}
\end{frame}

\begin{frame}[fragile]
  \frametitle{Свиване на вложени списъци}

  \tt{(deep-fold }<х-дъно> <в-дъно> <операция>\tt)

\pause

\begin{semiverbatim}
(define (deep-fold nv term op l)
  (cond ((null? l) \hzero{nv})
        ((atom? l) \vzero{(term l)})
        (else (op \vstep{(deep-fold nv term op (car l))}
                  \hstep{(deep-fold nv term op (cdr l))}))))
\end{semiverbatim}

\pause
\nxt{
\begin{semiverbatim}
(define (count-atoms l) (deep-fold \rvl0 \rvl{(lambda (x) 1)} \rvl+ l))
\end{semiverbatim}
\nxt{
\begin{semiverbatim}
(define (flatten l) (deep-fold \rvl{'()} \rvl{list} \rvl{append} l))
\end{semiverbatim}
\nxt{
\begin{semiverbatim}
\onslide<14>{(define (rcons x l) (append l (list x)))}

(define (deep-reverse l) (deep-fold \rvl{'()} \rvl{id} \rvl{rcons} l))
\end{semiverbatim}
}}}
\end{frame}

\section{Функции с произволен брой аргументи}

\begin{frame}
  \frametitle{Приемане на произволен брой аргументи}

  \begin{itemize}[<+->]
  \item \tta{(lambda }<списък> <тяло>\tta)
  \item създава функция с <тяло>, която получава <списък> от параметри
  \item \tta{(lambda (}\{<параметър>\}$^+$ \tta. <списък>\tta) <тяло>\tta)
  \item създава функция с <тяло>, която получава няколко задължителни <параметър> и <списък> от опционални параметри
  \item \tta{(define (}<функция> \tta. <списък>\tta) <тяло>\tta)
  \item еквивалентно на \\
    \tt{(define }<функция> \tt{(lambda }<списък> <тяло>\tt{))}
  \item \tta{(define (}<функция> \{<параметър>\}$^+$ \tta. <списък>\tta) <тяло>\tta)
  \item еквивалентно на \\
    \tt{(define }<функция>\\
    \hspace{10ex}\tt{(lambda (}\{<параметър>\}$^+$ \tt. <списък>\tt) <тяло>\tt{))}
  \end{itemize}
\end{frame}

\begin{frame}
  \frametitle{Примери}

  \begin{itemize}[<+->]
  \item \tt{(define (maximum x . l) (foldl1 max (cons x l)))}
  \item \evalstop{(maximum 7 3 10 2)}{10}
  \item \evalstop{(maximum 100)}{100}
  \item \evalstoerrp{(maximum)}
  \item \tt{(define (g x y . l) (append (append x l) (append y l)))}
  \item \evalstop{(g '(1 2 3) '(4 5 6)))}{(1 2 3 4 5 6)}
  \item \evalstop{(g '(1 2 3) '(4 5 6) 7 8))}{(1 2 3 7 8 4 5 6 7 8)}
  \end{itemize}
\end{frame}

\begin{frame}[fragile]
  \frametitle{Прилагане на функция над списък от параметри (\tt{apply})}

  \begin{itemize}[<+->]
  \item \tta{(apply }<функция> <списък>\tta)
  \item прилага <функция> над <списък> от параметри
  \item Примери:
  \item \evalsto{(apply + '(1 2 3 4 5))}{15}
  \item \evalstop{(apply append '((1 2) (3 4) (5 6)))}{(1 2 3 4 5 6)}
  \item \evalstop{(apply list '(1 2 3 4))}{(1 2 3 4)}
  \end{itemize}

  \pause
\small
\begin{verbatim}
(define (append . l)
   (cond ((null? l) '())
         ((null? (car l)) (apply append (cdr l)))
         (else (cons (caar l)
                     (apply append (cons (cdar l) (cdr l)))))))
\end{verbatim}
\end{frame}

\begin{frame}
  \frametitle{Оценяване на списък като комбинация (\tt{eval})}

  \begin{itemize}[<+->]
  \item \tta{(eval }<списък>\tta)
  \item връща оценката на \textbf{оценката на <списък>}
  \item Примери:
  \item \tt{(define a 2)}
  \item \evalsto a2
  \item \evalsto{(eval a)}2
  \item \evalstop{(eval 'a)}2
  \item \evalstop{(eval ''a)}a
  \item \evalstop{(eval (eval ''a))}2
  \item \evalstop{(eval (list '+ 5 7 a))}{14}
  \item \evalstoerrp{(eval (list 'define 'b 5))}
  \item \tt{(eval (list 'define ''b 5))} $\longleftrightarrow$ \tt{(define b 5)}
  \item {\small \evalstop{(eval (list 'if (list '< 2 5) (list 'quote 'a) 'b))}a}
  \item \tt{(define (apply f l) (eval (cons f l)))}
  \end{itemize}

  \pause
  \alert{Програмите на Scheme могат да се разглеждат като данни!}
\end{frame}

\end{document}
