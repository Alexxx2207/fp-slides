\documentclass{beamer}
\usepackage{fp}

\title{Списъци}

\date{28 октомври 2015 г.}

\begin{document}

\begin{frame}
  \titlepage
\end{frame}

\section{Точкови двойки}

\begin{frame}
  \frametitle{Точкови двойки}

  \begin{tabular}{cc}
    \hline
  \multicolumn 1{|c|}{\hspace{2ex}}&\multicolumn 1{c|}{\hspace{2ex}}\\
    \hline
  \bda&\bda\\
  \fbox3 &\fbox5
  \end{tabular}

  \vspace{1em}

  \begin{itemize}[<+->]
  \item \tta{(cons }<израз$_1$> <израз$_2$>\tta)
  \item Точкова двойка от оценките на <израз$_1$> и <израз$_2$>
  \item \tta{(car }<израз>\tta)
  \item \textbf{Първият} компонент на двойката, която е оценката на <израз>
  \item \tta{(cdr }<израз>\tta)
  \item \textbf{Вторият} компонент на двойката, която е оценката на <израз>
  \end{itemize}


\end{frame}

\begin{frame}
  \frametitle{Примери}
\end{frame}

\begin{frame}
  \frametitle{S-изрази}
\end{frame}

\begin{frame}
  \frametitle{All you need is $\lambda$ --- точкови двойки}
\end{frame}

\section{Списъци}

\begin{frame}
  \frametitle{Списъци в Scheme}
\end{frame}

\begin{frame}
  \frametitle{Вградени функции за списъци}
\end{frame}

\begin{frame}
  \frametitle{Форми на равенство в Scheme}
\end{frame}

\begin{frame}
  \frametitle{memq, \ldots}
\end{frame}

\begin{frame}
  \frametitle{Съкратени форми на \tt{car} и \tt{cdr}}
\end{frame}


\subsection{Рекурсия над списъци}


\begin{frame}
  \frametitle{Обхождане на списъци}
\end{frame}

\begin{frame}
  \frametitle{Конструиране на списъци}
\end{frame}

\section{Вложени списъци}

\begin{frame}
  \frametitle{Работа с вложени списъци}
\end{frame}

\begin{frame}
  \frametitle{Примери}
\end{frame}

\section{Функции от по-висок ред за списъци}

\begin{frame}
  \frametitle{\tt{map}}
\end{frame}

\begin{frame}
  \frametitle{\tt{filter}}
\end{frame}

\begin{frame}
  \frametitle{\tt{accumulate}}
\end{frame}

\begin{frame}
  \frametitle{\tt{accumulate-i}}
\end{frame}

\begin{frame}
  \frametitle{\tt{accumulate1}}
\end{frame}


\section{Функции с произволен брой аргументи}

\begin{frame}
  \frametitle{Дефиниция}
\end{frame}

\begin{frame}
  \frametitle{\tt{apply}}
\end{frame}

\begin{frame}
  \frametitle{\tt{eval}}
\end{frame}

\end{document}
