\documentclass{beamer}
\usepackage{fp}

\title[Среди и процеси]{Модел на средите и изчислителни процеси}

\date{14 октомври 2015 г.}

\begin{document}

\begin{frame}
  \titlepage
\end{frame}

\section{Модел на средите}

\begin{frame}
  \frametitle{Среди в Scheme}

  \begin{itemize}[<+->]
  \item Връзката между символите и техните оценки се записват в речник, който се нарича \textbf{среда}.
  \item Всеки символ има най-много една оценка в дадена среда.
  \item В даден момент могат да съществуват много среди, но само една от която е активна.
  \item \alert{Един символ може да има различни оценки в различни среди.}
  \item При стартиране Scheme по подразбиране работи в \textbf{глобалната среда}.
  \item В глобалната среда са дефинирани символи за стандартни операции и функции.
  \end{itemize}
\end{frame}

\begin{frame}
  \frametitle{Пример за среда}

  \begin{columns}[t,onlytextwidth]
    \column{0.5\textwidth}
    {}

    \begin{itemize}[<+->]
    \item \evalstoerr r
    \item \tt{(define r 5)}
    \item \evalsto{(+ r 3)}8
    \item \tt{(define (sq x) (* x x))}
    \item \evalsto{(sq 3)}9
    \item \evalsto{(sq r)}{25}
    \end{itemize}

    \column{0.5\textwidth}
    {}

    \begin{tabular}{|l@{\hskip 1ex}l@{\hskip 1ex}l|}
      \hline
      \textbf{E}&&
      \only<2->{\\\tt r&:&\tt 5}
      \only<4->{\\\tt{sq}&:&\funcenv x{(* x x)}E}
      \\[1.5em]\hline
    \end{tabular}
  \end{columns}
\end{frame}

\begin{frame}
  \frametitle{Функции и среди}

  \begin{itemize}[<+->]
  \item Всяка функция \tt f пази указател към средата \textbf E, в която е дефинирана.
  \item При извикване на \tt f:
    \begin{itemize}
    \item създава се нова среда \textbf{E$_1$}, която разширява \textbf{E}
    \item в \textbf{E$_1$} всеки символ означаващ формален параметър се свързва с оценката на фактическия параметър
    \item тялото на $f$ се изпълнява в \textbf{E$_1$}
    \end{itemize}
  \end{itemize}
\end{frame}

\begin{frame}
  \frametitle{Дърво от среди}
  \begin{itemize}[<+->]
  \item Всяка среда пази указател към своя ``родителска среда'', която разширява
  \item така се получава дърво от среди
  \item при оценка на символ в дадена среда \textbf E
    \begin{itemize}
    \item първо се търси оценката му в \textbf E
    \item ако символът не е дефиниран в \textbf E, се преминава към родителската среда
    \item при достигане на най-горната среда, ако символът не е дефиниран и в нея се извежда съобщение за грешка
    \end{itemize}
  \end{itemize}
\end{frame}

\begin{frame}
  \frametitle{Извикване на дефинирана функция}

  \begin{columns}[t,onlytextwidth]
    \column{0.5\textwidth}
    {}

    \begin{itemize}[<+->]
    \item \tt{(define r 5)}
    \item \tt{(define (sq x) (* x x))}
    \item \begin{tabular}[t]{lc}
            \{\textbf E\} &\tt{(sq r)}\\
            \nxt{&\bda\\
            \{\textbf E\} &\tt{(sq 5)}\\
            \nxt{&\bda\\
            \{\textbf{E$_1$}\} &\tt{(* x x)}\\
            \nxt{&\bda\\
            \{\textbf{E$_1$}\} &\tt{(* 5 5)}\\
            \nxt{&\bda\\
            &\tt{25}}}}}
          \end{tabular}
        \end{itemize}

    \column{0.5\textwidth}
    {}

    \begin{tabular}{c}      
    \begin{tabular}{|l@{\hskip 1ex}l@{\hskip 1ex}l|}
      \hline
      \textbf{E}&&
      \only<1->{\\\tt r&:&\tt 5}
      \only<2->{\\\tt{sq}&:&\funcenv x{(* x x)}E}
      \\[1.5em]\hline
    \end{tabular}\\
      \only<5->{
      \Bigg\uparrow\\
    \begin{tabular}{|l@{\hskip 1ex}l@{\hskip 1ex}l|}
      \hline
      \textbf{E$_1$}&&\\
%      \hline&&\\[-1em]\hline
      x&:&5
      \\\hline
    \end{tabular}}
    \end{tabular}
  \end{columns} 
\end{frame}

\end{document}
