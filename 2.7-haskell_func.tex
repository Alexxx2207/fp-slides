\documentclass[alsotrans]{beamerswitch}
\usepackage{fprog}

\title{Функтори и монади}

\date{10 февруари 2020 г.}

\lstset{language=Haskell,style=Haskell}

\begin{document}

\begin{frame}
  \titlepage
\end{frame}

\section{Функтори}

\begin{frame}[fragile]
  \frametitle{Класове от по-висок ред}

  \begin{itemize}[<+->]
  \item Досега разглеждахме \emph{класове} от типове, които имат
    сходно поведение (\lst{Eq}, \lst{Read}, \lst{Show}, \lst{Enum},
    \lst{Measurable}, \lst{Num}, \ldots).
  \item
    Разглеждахме и \emph{типови конструктори}, които позволяват дефиниране на параметризирани (генерични) типове (\lst{Maybe}, \lst{[]}, \lst{BinTree}, \lst{Tree}, \lst{IO}, \ldots).
  \item
    Нека да разгледаме \emph{клас от типови конструктори}, които имат някаква обща характеристика.
  \item
    \textbf{Пример:} Има ли нещо общо, което можем да правим с \lst{[]}, \lst{BinTree} и \lst{Tree}?
  \item Нещо, което не зависи от \emph{типа} на елементите в тези контейнери?
  \end{itemize}
\end{frame}

\begin{frame}[fragile]
  \frametitle{Примери за класове от конструктори}
  \begin{itemize}[<+->]
  \item
    \textbf{Пример:} Има ли нещо общо, което можем да правим с \lst{[]}, \lst{BinTree} и \lst{Tree}?
  \item Можем да намираме брой елементи
\begin{lstlisting}
class Countable c where
  count :: c a -> Integer
\end{lstlisting}
  \item Можем да намерим списък от всички елементи
\begin{lstlisting}
class Listable c where
  elements :: c a -> [a]
\end{lstlisting}
  \item Можем да приложим функция над всеки елемент
\begin{lstlisting}
class Functor f where
  fmap :: (a -> b) -> f a -> f b
\end{lstlisting}
\end{itemize}
\end{frame}

\begin{frame}
  \frametitle{Функтори}
  \begin{definition}
    Класът \lst{Functor} в Haskell се състои от типовите конструктори $f$, за които може да се дефинира \lst{fmap :: (a -> b) -> f a -> f b}.
  \end{definition}
  \pause
  За удобство операцията \lst!<$>! е инфиксен вариант на \lst{fmap}.\\ % $ за поправка на оцветяването
  \pause
  \textbf{Примери за функтори:}
  \begin{itemize}[<+->]
  \item \lst{Maybe}
  \item \lst{(,) a}  
  \item \lst{Either a}
  \item \lst{[]}
  \item \lst{BinTree}
  \item \lst{Tree}
  \item \lst{(->) r}
  \item \lst{IO}
  \end{itemize}
\end{frame}

\begin{frame}[fragile]
  \frametitle{Странни функторни екземпляри}
  \textbf{Пример:} да разгледаме екземпляра
\begin{lstlisting}
data Pill a = BluePill a | RedPill a
instance Functor Pill where
  fmap f (BluePill x) = RedPill (f x)
  fmap f (RedPill  x) = BluePill (f x)
\end{lstlisting}
  \pause
  \textbf{Проблем №1:}
  \begin{itemize}[<+->]
  \item \lst{fmap id (BluePill 2) = RedPill 2}
  \item \lst{fmap} с ``празна'' функция променя структурата на функтора!
  \end{itemize}
  \onslide<+->
  \textbf{Проблем №2:}
  \begin{itemize}[<+->]
  \item \lst{fmap (+3) (BluePill 3) = RedPill 6}
  \item \lst{fmap (+1) (fmap (+2) (BluePill 3)) = BluePill 6}
  \item Има значение колко поред функции ще приложим!
  \end{itemize}
\end{frame}

\begin{frame}
  \frametitle{Функторни закони}
  \begin{definition}
    \emph{Функтор} наричаме екземпляр на класа \lst{Functor} такъв, че:
    \begin{enumerate}
    \item \lst{fmap id} \eqv \lst{id} (запазване на идентитета)
    \item \lst{fmap f . fmap g} \eqv \lst{fmap (f . g)} (дистрибутивност относно композиция)
    \end{enumerate}
  \end{definition}
  \pause
  Функторните закони ни дават гаранция, че реализацията на \lst{fmap} е ``неутрална'' към функтора и променя стойностите в него само и единствено на базата на подадената функция \lst{f}.\\
  \pause
  Всички примерни екземпляри (освен \lst{Pill}) удовлетворяват функторните закони.\\
  \pause
  Можем да мислим, че \lst{fmap} ``повдига'' функцията \lst{f} от елементи към функтори.
\end{frame}

\section{Апликативни функтори}

\begin{frame}
  \frametitle{\tt{fmap} с двуаргументни функции}
  Можем ли да използваме \tt{fmap} за ``повдигане'' на двуаргументна функция?\\
  \pause
  \onslide<+->
  \textbf{Пример:} \evalstoerrp{fmap (+) (Just 3) (Just 5)}\\
  \onslide<+->
  \textbf{Проблем:} \typestop{fmap (+) (Just 3)}{Maybe (Int -> Int)}\\
  \onslide<+->
  Получаваме функтор над функция, която не можем директно да приложим над функтор над стойност!\\
  \onslide<+->
  \textbf{Идея:} Да разбием \lst{fmap} на две части:
  \begin{itemize}[<+->]
  \item повдигане на функтор над функция към функция над функтори
    \begin{itemize}
    \item \lst{f (a -> b) -> f a -> f b}
    \end{itemize}
  \item повдигане на обикновена функция към функтор над функция
    \begin{itemize}
    \item \lst{(a -> b) -> f (a -> b)} 
    \end{itemize}
  \end{itemize}
  \onslide<+->
  Функторите, които поддържат такова разлагане на \lst{fmap} наричаме \emph{апликативни}.
\end{frame}

\begin{frame}[fragile]
  \frametitle{Класът \tt{Applicative}}
\begin{lstlisting}
class (Functor f) => Applicative f where
  pure  :: a -> f a
  (<*>) :: f (a -> b) -> f a -> f b
\end{lstlisting}
  \onslide<+->
  Можем да дефинираме \alt<+->{\alt<+->{\alt<+->{\lst{fmap = (<*>) . pure}}{\lst{fmap f = (<*>) (pure f)}}}{\lst{fmap f a = (<*>) (pure f) a}}}{\lst{fmap f a = pure f <*> a}}.\\
  \onslide<+->
  \textbf{Примери за апликативни функтори:}
  \begin{itemize}[<+->]
  \item \lst{Maybe}
  \item \lst{Either a}
  \item \lst{[]}
  \item \lst{ZipList}
  \item \lst{(->) r}
  \item \lst{IO}
  \end{itemize}
\end{frame}

\begin{frame}
  \frametitle{Функции за апликативни функтори}
  \begin{itemize}[<+->]
  \item \lst!liftA2 :: (Applicative f) => !\\
    \hspace{10ex}\lst{(a -> b -> c) -> f a -> f b -> f c}
    \begin{itemize}
    \item повдига двуаргументна функция над функтор
    \item \lst!liftA2 f a b = f <$> a <*> b!   % $ оцветяване на синтаксиса
    \item \textbf{Пример:}\\\evalstop{liftA2 (+) [2,3] [10,20,30]}{[12,22,32,13,23,33]}
    \end{itemize}
  \item \lst{sequenceA :: (Applicative f) => [f a] -> f [a]}
    \begin{itemize}
    \item повдига списък от функтори до функтор над списък
    \item     \lst{sequenceA []     = pure []}
    \item<.-> \lst{sequenceA (x:xs) = liftA2 (:) x (sequenceA xs)}
    \item     \lst{sequenceA = foldr (liftA2 (:)) (pure [])}
    \item \textbf{Пример:}\\\evalstop{sequenceA [Just 2, Just 3, Just 5]}{Just [2,3,5]}
    \item \textbf{Пример:} \evalstop{sequenceA [Just 2, Nothing, Just 5]}{Nothing}
    \end{itemize}
  \end{itemize}
\end{frame}

\begin{frame}
  \frametitle{Закони за апликативни функтори}
  \begin{definition}
    \emph{Апликативен функтор} наричаме екземпляр на класа \lst{Applicative}, за който:
    \begin{enumerate}[<+->]
    \item \lst{pure f <*> x} \eqv \lst{fmap f x}
    \item \lst{pure id <*> v} \eqv \lst{v}
    \item \lst{pure (.) <*> u <*> v <*> w} \eqv \lst{u <*> (v <*> w)}
    \item \lst{pure f <*> pure x} \eqv \lst{pure (f x)}
    \item \lst{u <*> pure y} \eqv \lst{pure ($ y) <*> u} % $ оцветяване на синтаксис
    \end{enumerate}
  \end{definition}
\end{frame}

\section{Монади}

\begin{frame}
  \frametitle{Операцията ``свързване'' (bind)}
  
  \begin{itemize}[<+->]
  \item Функторите ни позволяваха да превърнем \emph{функция над елементи} във функция над функтори:
    \begin{itemize}
    \item \evalsto{(+3) <$> [1,2]}{[4,5]} % $ оцветяване на синтаксиса
    \end{itemize}
  \item Апликативните функтори ни позволяваха да превърнем \emph{функтор над функция} към функция над функтори
    \begin{itemize}
    \item \evalsto{(+) <$> [1,2] <*> [10,20]}{[11,12,21,22]} % $ оцветяване на синтаксиса
    \end{itemize}
  \item Но как можем да превърнем \emph{функция, връщаща функтор} във функция над функтори?
    \begin{itemize}
    \item \evalsto{(\\x -> [1..x]) =<< [3,4]}{[1,2,3,1,2,3,4]} % >> оцветяване на синтаксиса
    \item Искаме структурата на функтора-резултат да може да зависи от стойността във функтора-параметър!
    \item \lst!(=<<) :: (a -> f b) -> f a -> f b!  % >> оцветяване на синтаксиса
    \item По-често се използва операцията ``свързване'' (bind) с разменени аргументи:
    \item \lst!(>>=) :: f a -> (a -> f b) -> f b!
    \end{itemize}
  \end{itemize}
\end{frame}

\begin{frame}[fragile]
  \frametitle{Класът \tt{Monad}}
\begin{lstlisting}
class (Applicative m) => Monad m where  
  return :: a -> m a  
  return = pure     
  
  (>>=) :: m a -> (a -> m b) -> m b
  (>>)  :: m a -> m b -> m b  
  x >> y   =   x >>= \_ -> y  
\end{lstlisting}
  \pause
  \textbf{Примери за монади:}
  \begin{itemize}[<+->]
  \item \lst{Maybe}
  \item \lst{[]}
  \item \lst{(->) r}
  \item \lst{IO}
  \end{itemize}
  \onslide<+->
  Синтаксисът \lst{do} работи за всички екземпляри на \lst{Monad}!
\end{frame}

\begin{frame}[fragile]
  \frametitle{Монадни функции (1)}
  \begin{itemize}[<+->]
  \item \lst{liftM :: (Monad m) => (a -> b) -> m a -> m b}
    \begin{itemize}
    \item \lst{fmap} за монади
    \item \lst!liftM f m = m >>= (\x -> return $ f x)! % $ оцветяване на синтаксиса
    \end{itemize}
  \item \lst{ap :: (Monad m) => m (a -> b) -> m a -> m b}
    \begin{itemize}
    \item \lst{<*>} за монади
    \item \lst!ap mf m = mf >>= (\f -> m >>= (\x -> return $ f x))! % $ оцветяване на синтаксиса
    \end{itemize}
  \item \lst{liftM2::(Monad m) => (a -> b -> c) -> m a -> m b -> m c}
    \begin{itemize}
    \item \lst{liftA2} за монади
\onslide<+->
\begin{lstlisting}
liftM2 f m1 m2 = m1 <<= (\x1 ->
                 m2 <<= (\x2 ->
                 return $ f x1 x2))
\end{lstlisting}
    \end{itemize}
  \end{itemize}
\end{frame}

\begin{frame}[fragile]
  \frametitle{Монадни функции (2)}
  \begin{fixedarea}
    \begin{itemize}[<+->]
    \item \lst{join :: (Monad m) => m (m a) -> m a}
      \begin{itemize}
      \item ``слива'' двойната опаковка в единична
      \item \alt<+->{\alt<+->{\lst{join = (>>= id)}}{\lst{join mm = (>>= id) mm}}}{\lst{join mm = mm >>= id}}
      \item Можем да дефинираме \lst{(>>=)} чрез \lst{join} и \lst{fmap}:
      \item<.-> \lst{m >>= f = join (fmap f m)} 
      \end{itemize}
    \item \lst{filterM :: (Monad m) => (a -> m Bool) -> [a] -> m [a]}
      \begin{itemize}
      \item Филтрира с предикат, връщащ ``опаковани'' булеви стойности
      \item Резултатът е ``опакованите'' елементи на списъка
      \item \lst{powerset = filterM (\x -> [True,False])}
      \end{itemize}
    \item \lst{foldM :: (Monad m) => (a -> b -> m a) -> a -> [b] -> m a}
      \begin{itemize}
      \item Натрупва елементи от списък с монадна операция
      \item Натрупването е ляво (итеративен процес, подобно на \lst{foldl})
      \item
        \lst!boundSum lim = foldM (\x y -> if x+y < lim!\\
        \hspace{30ex}\lst{then Just (x+y) else Nothing) 0}
      \item \evalstop{boundSum 60 [1..10]}{Just 55}
      \item \evalstop{boundSum 50 [1..10]}{Nothing}
      \end{itemize}
    \end{itemize}
  \end{fixedarea}
\end{frame}

\begin{frame}[fragile]
  \frametitle{Монадни закони}
  \begin{definition}
    \emph{Монада} наричаме инстанция на класа \lst{Monad}, за която:
    \small
    \begin{enumerate}[<+->]
    \item \lst{return x >>= f} \eqv \lst{f x} (ляв идентитет)
    \item \lst{m >>= return}   \eqv \lst{m}   (десен идентитет)
    \item \lst{(m >>= f) >>= g} \eqv \lst{m >>= (\x -> f x >>= g)} (асоциативност)
    \end{enumerate}
  \end{definition}
  \onslide<+->
  Композиция на монадни функции:\\
\begin{lstlisting}
(<=<) :: (Monad m) => (b -> m c) -> (a -> m b) -> (a -> m c)  
f <=< g   =   \x -> g x >>= f
\end{lstlisting}
  \onslide<+->
  Монадните закони чрез композиция:
  \begin{enumerate}
  \item \lst{f <=< return} \eqv \lst{f} (ляв идентитет)
  \item \lst{return <=< f} \eqv \lst{f} (десен идентитет)
  \item \lst{f <=< (g <=< h)} \eqv \lst{(f <=< g) <=< h} (асоциативност)
  \end{enumerate}
\end{frame}
\end{document}

% LocalWords:  BinTree BluePill RedPill ZipList liftA sequenceA xs ap
% LocalWords:  liftM mf filterM foldM boundSum lim
