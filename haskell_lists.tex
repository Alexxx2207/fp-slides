\documentclass{beamer}
\usepackage{fprog}

\title{Кортежи и списъци}

\date{6--7 януари 2016 г.}

\begin{document}

\begin{frame}
  \titlepage
\end{frame}

%\includeonlyframes{current}

\section{Кортежи}

\begin{frame}
  \frametitle{Кортежи (tuples)}
  Кортежите са наредени $n$-торки от данни от произволен тип.
  \begin{itemize}[<+->]
  \item Примери: \tt{(1, 2)}, \tt{(3.5,'A', False)}, \tt{(("square"{}, (\^{}2)), 1.0)}
  \item Тип кортеж от $n$ елемента: \tt{(t$_1$, t$_2$, \ldots, t$_n$)}
  \item Допустими стойности: наредени $n$-торки от вида \tt{(x$_1$, x$_2$, \ldots, x$_n$)}, където \tt{x$_i$} е от тип \tt{t$_i$}
  \item Позволяват ``пакетиране"' на няколко стойности в една
  \item Операции за наредени двойки:
    \begin{itemize}
    \item \tt{(,) :: a -> b -> (a,b)} --- конструиране на наредена двойка
    \item \tt{fst :: (a,b) -> a} --- първа компонента на наредена двойка
    \item \tt{snd :: (a,b) -> b} --- втора компонента на наредена двойка
    \end{itemize}
  \end{itemize}
\end{frame}

\begin{frame}
  \frametitle{Потребителски типове}
  \begin{itemize}[<+->]
  \item Типът \tt{(String, Int)} може да означава:
    \begin{itemize}
    \item име и ЕГН на човек
    \item продукт с описание и количество
    \item сонет на Шекспир и поредния му номер 
    \end{itemize}
  \item Удобно е да именуваме типовете, за да означаваме смисъла им
  \item \tta{type }<конструктор> \tta= <тип>
    \begin{itemize}
    \item конструкторите са идентификатори, започващи с главна буква
    \end{itemize}
  \item Примери:
    \begin{itemize}
    \item \tt{type Student = (String, Int, Double)}
    \item \tt{type Point = (Double, Double)}
    \item \tt{type Triangle = (Point, Point, Point)}
    \item \tt{type Translation = Point -> Point}
    \item \tt{type Vector = Point}
    \item \tt{addVectors :: Vector -> Vector -> Vector}
    \item<.-> \tt{addVectors v1 v2 = (fst v1 + fst v2, snd v1 + snd v2)}
    \end{itemize}
  \end{itemize}
\end{frame}

\begin{frame}
  \frametitle{Особености на кортежите}
  \begin{itemize}[<+->]
  \item \evalstoerrp{fst (1,2,3)}
    \begin{itemize}
    \item \tt{fst} и \tt{snd} работят само над наредени двойки!
    \end{itemize}
  \item \tt{((a,b),c)} $\neq$ \tt{(a,(b,c))} $\neq$ \tt{(a,b,c)}
  \item Няма специален тип кортеж от един елемент\ldots
  \item \ldots но има тип ``празен кортеж'' \tt{()} с единствен елемент \tt{()}
    \begin{itemize}
    \item в други езици такъв тип се нарича \tt{unit}
    \item използва се за означаване на липса на информация
    \end{itemize}
  \end{itemize}
\end{frame}

\begin{frame}[fragile]
  \frametitle{Образци на кортежи}
  Образец на кортеж е конструкция от вида \tt{(p$_1$, p$_2$, \ldots, p$_n$)}, където \tt{p$_i$} са образци.
  \begin{itemize}[<+->]
  \item \verb#addVectors (x1, y1) (x2, y2) = (x1 + x2, y1 + y2)#
  \item \verb#fst (x,_) = x#
  \item<.-> \verb#snd (_,y) = y#
  \item \verb#getFN :: Student -> Int#
  \item<.-> \verb#getFN (_, fn, _) = fn#
  \item образците на кортежи могат да се използват за ``разглобяване'' на кортежи при дефиниция
  \item \verb#(x,y) = (3.5, 7.8)#
  \item \verb#let (_, fn, grade) = student in (fn, min (grade + 1) 6)#
  \end{itemize}
\end{frame}

\begin{frame}[fragile]
  \frametitle{Именувани образци}
  \begin{itemize}[<+->]
  \item намиране на студент с по-висока оценка
\begin{semiverbatim}
betterStudent (name1, fn1, grade1) (name2, fn2, grade2)
 | grade1 > grade2 = (name1, fn1, grade1)
 | otherwise       = (name2, fn2, grade2)
\end{semiverbatim}
  \item ами ако имахме 10 полета?
  \item удобно е да използваме \alert{именувани образци}
  \item{} <име>\tta@<образец> \onslide<+->
\begin{semiverbatim}
betterStudent s1@(_, _, grade1) s2@(_, _, grade2)
 | grade1 > grade2 = s1
 | otherwise       = s2
\end{semiverbatim}
  \end{itemize}
\end{frame}

\section{Списъци}

\subsection{Дефиниция и синтаксис}


\begin{frame}
  \frametitle{Списъци}
  
\end{frame}

\begin{frame}
  \frametitle{Синтаксис за списъци}
  
\end{frame}

\begin{frame}
  \frametitle{Генератори на списъци}
  
\end{frame}

\begin{frame}
  \frametitle{Основни функции за списъци}
  % head, tail, null, length
\end{frame}

\begin{frame}
  \frametitle{Низове}
  
\end{frame}

\subsection{Работа със списъци}

\begin{frame}
  \frametitle{Рекурсия над списъци}
  %++, !!, reverse, elem, notElem
\end{frame}

\begin{frame}
  \frametitle{Образци и списъци}
  
\end{frame}

\begin{frame}
  \frametitle{Полиморфни функции}
\end{frame}

\begin{frame}
  \frametitle{Класове от типове (typeclasses)}
  
\end{frame}

\begin{frame}
  \frametitle{Стандартни класове}
  
\end{frame}

\subsection{Отделяне на списъци}

\begin{frame}
  \frametitle{Отделяне на списъци (list comprehension)}
  
\end{frame}

\begin{frame}
  \frametitle{Отделяне от повече списъци}
  
\end{frame}

\section{Функции над списъци}

\begin{frame}
  \frametitle{Отрязване на списъци}
  % take, drop, splitAt
\end{frame}

\begin{frame}
  \frametitle{Агрегиращи функции}
  % maximum, minimum, product, sum, and, or, concat
\end{frame}

\section{Функции от по-висок ред над списъци}

\begin{frame}
  \frametitle{Трансформация (\tt{map})}
  
\end{frame}

\begin{frame}
  \frametitle{Филтриране (\tt{filter})}
  
\end{frame}

\begin{frame}
  \frametitle{Дясно и ляво свиване (\tt{foldr} и \tt{foldl})}
  
\end{frame}

\begin{frame}
  \frametitle{Свиване на непразни списъци (\tt{foldr1} и \tt{foldl1})}
  
\end{frame}

\begin{frame}
  \frametitle{Сканиране на списъци (\tt{scanl}, \tt{scanr})}
  
\end{frame}


\begin{frame}
  \frametitle{Съшиване на списъци (\tt{zip}, \tt{zipWith})}
  
\end{frame}

\begin{frame}
  \frametitle{Разбиване на списъци (\tt{break}, \tt{span}, \tt{takeWhile}, \tt{dropWhile})} 
  
\end{frame}

\begin{frame}
  \frametitle{Логически квантори (\tt{all}, \tt{any})}
  
\end{frame}

\end{document}
