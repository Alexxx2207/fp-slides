\documentclass{beamer}
\usepackage{fp}

\title{Функции от по-висок ред}

\date{28 октомври 2015 г.}

\begin{document}

\begin{frame}
  \titlepage
\end{frame}

\section{Функциите като параметри}

\begin{frame}[fragile]
  \frametitle{Подаване на функции като параметри}

  В Scheme функциите са ``първокласни'' стойности.
  \vspace{1em}

  \pause

  Примери:
  \begin{itemize}[<+->]
  \item \verb#(define (fixed-point? f x) (= (f x) x))#
  \item \evalsto{(fixed-point? sin 0)}{\#t}
  \item \evalsto{(fixed-point? exp 1)}{\#f}
  \item \evalstoerr{(fixed-point? + 0)}
  \item \alt<11>{\tt{(define (branch p? f g x) ((if (p? x) f g) x))}}
    {\tt{(define (branch p? f g x) (if (p? x) (f x) (g x)))}}
  \item \evalsto{(branch odd? exp fact 4)}{24}
  \item \verb#(define (id x) x)#
  \item \evalsto{(branch number? log id "1")}{"1"}
  \item \evalsto{(branch string? number? procedure? symbol?)}{\#t}
  \end{itemize}

\end{frame}

\begin{frame}
  \frametitle{Функции от по-висок ред}

  \begin{definition}
    Функция, която приема функция за параметър се нарича \emph{функция от по-висок ред}.
  \end{definition}

  \pause
  
  \begin{itemize}[<+->]
  \item \tt{fixed-point?} и \tt{branch} са функции от по-висок ред
  \item \alert{Примери за математически функции от по-висок ред?}
  \item Всички функции в $\lambda$-смятането са от по-висок ред!
  \end{itemize}
\end{frame}

\begin{frame}[fragile]
  \frametitle{Задачи за сумиране}

  \textbf{Задача:} Да се пресметнат следните суми:
  \begin{itemize}
  \item $k^2 + (k+1)^2 + \ldots + 100^2$ за $k \leq 100$
  \item $\int_a^b f(x) \approx \Delta x\big(f(a) + f(a+\Delta x) + f(a+2\Delta x) + \ldots + f(b)\big)$
  \item $x + e^x + e^{e^x} + e^{e^x} + \ldots$ докато поредното събираемо е $\leq 10^{10}$
  \end{itemize}

  \pause

  \newcommand{\upper}{\textcolor<5->{red}}
  \newcommand{\term}{\textcolor<6->{green}}
  \newcommand{\next}{\textcolor<7->{blue}}

\begin{semiverbatim}
(define (sum1 k)
  (if (> k \upper{100}) 0 (+ \term{(* k k)} (sum1 \next{(+ k 1)}))))
\end{semiverbatim}

  \pause

\begin{semiverbatim}
(define (sum2 a b f dx)
  (if (> a \upper{b}) 0 (+ \term{(* dx (f a))} (sum2 \next{(+ a dx)} b f dx))))
\end{semiverbatim}
  
  \pause

\begin{semiverbatim}
(define (sum3 x)
  (if (> x \upper{(expt 10 10)}) 0 (+ \term{x} (sum3 \next{(expt x)}))))
\end{semiverbatim}

\end{frame}

\begin{frame}[fragile]
  \frametitle{Обобщена функция за сумиране}
  Да се напише функция от по-висок ред \tt{sum}, която пресмята сумата:

  \begin{equation*}
    \sum_{\substack{i=a \\i \leftarrow next(i)}}^b term(i).
  \end{equation*}

\pause

\begin{semiverbatim}
(define (sum a \textcolor{red}b \textcolor{green}{term} \textcolor{blue}{next})
  (if (> a \textcolor{red}b) 0 (+ \textcolor{green}{(term a)} (sum \textcolor{blue}{(next a)} \textcolor{red}b \textcolor{green}{term} \textcolor{blue}{next}))))
\end{semiverbatim}
\end{frame}

\begin{frame}[fragile]
  \frametitle{Приложения на \tt{sum}}

  Решение на задачата чрез \tt{sum}:

  \pause
\begin{verbatim}
(define (square x) (* x x))
(define (1+ x) (+ x 1))
(define (sum1 k) (sum k 100 square 1+))
\end{verbatim}
\vspace{1em}

  \pause

\begin{semiverbatim}
(define (sum2 a b f dx)
\only<3>{  (define (term x) (* dx (f x)))
}  (define (next x) (+ x dx))
  \only<4>{(* dx }(sum a b \alt<3>{term}f next))\only<4>)
\end{semiverbatim}
\vspace{1em}

  \pause

\begin{verbatim}
(define (sum3 x) (sum x (expt 10 10) id expt))
\end{verbatim}
\end{frame}

\begin{frame}[fragile]
  \frametitle{Обобщена функция за произведение}
  Да се напише функция от по-висок ред \tt{product}, която пресмята:

  \begin{equation*}
    \prod_{\substack{i=a \\i \leftarrow next(i)}}^b term(i).
  \end{equation*}

\pause

\newcommand{\fb}{\only<4>\fbox}

\begin{semiverbatim}
(define (product a \textcolor{red}b \textcolor{green}{term} \textcolor{blue}{next})
  (if (> a \textcolor{red}b) \fb1 (\fb* \textcolor{green}{(term a)} (product \textcolor{blue}{(next a)} \textcolor{red}b \textcolor{green}{term} \textcolor{blue}{next}))))
\end{semiverbatim}

\pause

\begin{semiverbatim}
(define (sum a \textcolor{red}b \textcolor{green}{term} \textcolor{blue}{next})
  (if (> a \textcolor{red}b) \fb0 (\fb+ \textcolor{green}{(term a)} (sum \textcolor{blue}{(next a)} \textcolor{red}b \textcolor{green}{term} \textcolor{blue}{next}))))
\end{semiverbatim}

\end{frame}

\section{\tt{accumulate}}

\begin{frame}[fragile]
  \frametitle{Обобщена функция за натрупване}

  Да се напише функция, която пресмята

  \begin{equation*}
    term(a) \oplus \bigg(term\big(next(a)\big) \oplus \Big(\ldots \oplus \big(term(b) \oplus \bot\big) \ldots\Big)\bigg),
  \end{equation*}

  където $\oplus$ е бинарна операция,\\
  а $\bot$ е нейната ``(дясна) нулева стойност'', т.е. $x\oplus\bot = x$.

  \pause
\small
\begin{verbatim}
(define (accumulate op nv a b term next)
  (if (> a b) nv
      (op (term a) (accumulate op nv (next a) b term next))))
\end{verbatim}
\pause
\begin{verbatim}
(define (sum a b term next) (accumulate + 0 a b term next))
(define (product a b term next) (accumulate * 1 a b term next))
\end{verbatim}
\end{frame}

\begin{frame}[fragile]
  \frametitle{Задача: пресмятане на полином}

  Да се пресметне стойността на полинома

  \begin{equation*}
  P_n(x) = x^n + 2x^{n-1} + \ldots + (n-2)x^3 + (n-1)x^2 + nx + (n+1)
  \end{equation*}
  \vspace{1em}

  \pause

  \textbf{Решение №1:}
\begin{verbatim}
(define (polynome x n)
  (define (term i) (* (- (1+ n) i) (expt x i)))
  (accumulate + 0 0 n term 1+))
\end{verbatim}

  \pause
  
  \vspace{1em}
  \alert{Можем ли да решим задачата без да извикваме \tt{expt} на всяка стъпка?}
\end{frame}

\begin{frame}[fragile]
  \frametitle{Правило на Хорнер}

  \begin{equation*}
    P_n(x) = \Bigg(\bigg(\Big(\ldots\big((x + 2)x + 3\big)x + \ldots\Big)x + (n-1)\bigg)x + n\Bigg)x + (n+1)
  \end{equation*}

  \pause

  \alert{Можем ли да сметнем с \tt{accumulate}?}

  \pause

  \textbf{Идея:} Да използваме операцията $a \oplus b := ax + b$.

  \pause

  \alert{Коя е ``нулевата стойност''  $\bot$?}
  \vspace{1em}
  
  \pause

  \textbf{Решение №2:}
\begin{verbatim}
(define (p n x)
  (define (op a b) (+ (* a x) b))
  (accumulate op 0 0 n id 1+))
\end{verbatim}

  \pause
  
  \alert{Не смята правилно!}
\end{frame}

\begin{frame}[fragile]
  \frametitle{Правило на Хорнер}

Всъщност пресметнахме:
\begin{equation*}
  Q_n(x) = x + 2x + 3x + 4x + \ldots + nx = \frac{n(n+1)}2x.
\end{equation*}
  \pause
  \textbf{Идея:} Да използваме операцията $a \oplus b := a + bx$.
  \vspace{1em}
  \pause

  \textbf{Решение №2:}
\begin{verbatim}
(define (p n x)
  (define (op a b) (+ a (* b x)))
  (accumulate op 0 0 n id 1+))
\end{verbatim}

  \pause
  
  \alert{Пак не смята правилно!!!} 

\end{frame}

\begin{frame}
  \frametitle{Ляво и дясно натрупване}

  Всъщност пресметнахме:
  \begin{eqnarray*}
    R_n(x) &=& x\Bigg(1+x\bigg(2+x\Big(\ldots+x\Big((n-1)+x(n+x(n+1))\Big)\ldots\Big)\bigg)\Bigg)\\
          &=&  (n+1)x^{n+1} + nx^n + (n-1)x^{n-1} + \ldots + 3x^3 + 2x^2 + x
  \end{eqnarray*}
  вместо
  \begin{equation*}
    P_n(x) = \Bigg(\bigg(\Big(\ldots\big((x + 2)x + 3\big)x + \ldots\Big)x + (n-1)\bigg)x + n\Bigg)x + (n+1)
  \end{equation*}

  \pause

  \alert{За некомутативни и неасоциативни  операции $\oplus$ има значение в какъв ред са скобите!}
\end{frame}

\end{document}
