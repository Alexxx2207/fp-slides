\documentclass{beamer}
\usepackage{fprog}

\title{Организация на курса}

\date{3 октомври 2018 г.}

\begin{document}

\begin{frame}
  \titlepage
\end{frame}

\begin{frame}
  \frametitle{Екип}

  \begin{itemize}
  \item \textbf{1 група:} Слави Боянов
  \item \textbf{2 група:} Андрей Дренски
  \item \textbf{3 група:} Еслин Каранасуф
  \item \textbf{И група:} Антон Петков, Димитър Узунов
  \item \textbf{Практикум:} Георги Любенов
  \end{itemize}
\end{frame}

\begin{frame}
  \frametitle{Компоненти за оценяване}

  \begin{itemize}
  \item Контролни: 2 $\times$ 30 т.
  \item Проект: 1 $\times$ 60 т. максимум
  \item Блиц тестове: 5 $\times$ 2 т. \alert{(минимум 5 т.)}
  \item Бонус от упражнения: 1 $\times$ 10 т.
  \item Изпит задачи: 1 $\times$ 50 т. \alert{(минимум 20 т.)}
  \item Изпит теория: 1 $\times$ 50 т. \alert{(минимум 20 т.)}
  \item \textbf{Максимум:} 240 т.
  \item \textbf{Максимум текущ контрол:} 140 т.
  \end{itemize}
\end{frame}

\begin{frame}
  \frametitle{Схема за оценяване}

  \begin{itemize}
  \item 180 т. = Отличен 6.00
  \item 100 т. = освобождаване от изпит на задачи с 20 т.
  \item 60 т. = Среден 3.00
  \end{itemize}
\end{frame}

\begin{frame}
  \frametitle{Оценяване по ФП-практикум}

  \begin{itemize}
  \item Домашни: 7 $\times$ 10 т.
  \item Проект: 1 $\times$ 60 т. максимум \alert{(минимум 25 т.)}
  \item \textbf{Максимум:} 130 т.
  \item 110 т. = Отличен 6.00
  \item 50 т. = Среден 3.00
  \end{itemize}
\end{frame}

\begin{frame}
  \frametitle{Learn.fmi}
  
  \begin{itemize}
  \item \url{https://learn.fmi.uni-sofia.bg/}
  \item Бакалаври, зимен семестър 2018/2019
    \begin{itemize}
    \item И
      \begin{itemize}
      \item Функционално програмиране (И) 2018/2019
      \end{itemize}
    \end{itemize}
  \end{itemize}
\end{frame}

\end{document}
